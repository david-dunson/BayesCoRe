%&latex
\documentclass[10pt,fleqn]{article}
%\pdfoutput=1

\addtolength{\oddsidemargin}{-.875in}
\addtolength{\evensidemargin}{-.875in}
\addtolength{\textwidth}{1.75in}

\addtolength{\topmargin}{-.875in}
\addtolength{\textheight}{1.75in}

\openup 1em

%macro for commenting
\usepackage{color}
\newcommand{\leo}[1]{{\color{blue}{Leo: #1}}}

% \newcommand{\Xbeta}{ X_i \theta}
\newcommand{\xbeta}{ x_i \beta}
\newcommand{\xtheta}{ x_i \theta}
% \newcommand{\xbetaij}{ x_{ij}^T \theta}
\newcommand{\sgamma}{s_{ij}^T\gamma_i}

\usepackage[round]{natbib}

\usepackage{rotating}
\usepackage{graphicx}
\usepackage{subcaption}

\usepackage{float}
\usepackage{bbm}

\usepackage{amsthm,amsmath, amssymb} 
\usepackage{mathrsfs}
\usepackage{subcaption}
\usepackage{nicefrac}

\usepackage{xcolor}
\newcommand{\aki}[1]{\textcolor{red}{Aki: #1}}

\newtheorem{theorem}{Theorem}
\newtheorem{lemma}{Lemma}
\newtheorem{corollary}{Corollary}
\newtheorem{remark}{Remark}
\newtheorem{example}{Example}


\usepackage{algorithm}
\usepackage{algpseudocode}

%\usepackage{mhequ}
\newcommand{\be}{\begin{equation}\begin{aligned}}
\newcommand{\ee}{\end{aligned}\end{equation}}
\newcommand{\bb}[1]{\mathbb{#1}}
\newcommand{\mc}[1]{\mathcal{#1}}
\DeclareMathOperator{\Binom}{Binomial}
\DeclareMathOperator{\No}{No}
\DeclareMathOperator{\PG}{PG}
\DeclareMathOperator{\IG}{Inverse-Gamma}
\DeclareMathOperator{\Ga}{Gamma}
\DeclareMathOperator{\Bern}{Bernoulli}
\DeclareMathOperator{\U}{Uniform}
\DeclareMathOperator{\Poi}{Poisson}
\DeclareMathOperator{\NB}{NB}
\DeclareMathOperator{\cov}{cov}
\DeclareMathOperator{\var}{var}
\DeclareMathOperator{\diag}{diag}
\DeclareMathOperator{\Diag}{Diag}
\newcommand{\KL}[2]{\textnormal{KL}\left(#1 \parallel #2\right)}
\DeclareMathOperator{\1}{\mathbbm{1}}
\DeclareMathOperator{\bigO}{\mc O}
\newcommand{\dt}{\epsilon} % Stepsize of leapfrog
\newcommand{\mass}{M} % Mass matrix
\newcommand{\hess}{\mathbf{H}} % Hessian notation.



\thispagestyle{empty}
\baselineskip=28pt

\title{\textbf{Constraint Relaxation for Bayesian Modeling with Parameter Constraints}}
\author{Leo Duan, Akihiko Nishimura,  Alex Young,  David Dunson}
\date{}
\begin{document}

\maketitle
{\bf Abstract:} Prior information often takes  the form of parameter constraints. Bayesian methods include such information through prior distributions having constrained support. By using posterior sampling algorithms, one can quantify uncertainty without relying on asymptotic approximations. However, outside of narrow settings, parameter constraints
make it difficult to develop new  prior and/or   efficient posterior sampling algorithms. In this work, we first describe a general approach to utilize
the large pool of unconstrained distributions in constrained space,  then we propose to relax the parameter support  into the neighborhood surrounding constrained
space for convenient posterior estimation. The constraint relaxation can
be done using data augmentation technique or    with an approximation function. General off the shelf posterior sampling algorithms, such as Hamiltonian Monte Carlo (HMC), can then be used directly. We illustrate this approach through multiple examples involving equality and inequality constraints. While existing methods tend to rely on conjugate families or sophisticated reparameterization, our proposed approach frees us up to define new classes of  models for constrained problems. We illustrate this through application to a variety of simulated and real datasets.
\vskip 12pt
%\baselineskip=12pt
%\par\vfill\noindent
{\noindent KEY WORDS: Simplex,  Stiefel Manifold, Parameter Expansion}
%\par\medskip\noindent
%\clearpage\pagebreak\newpage
\pagenumbering{arabic}


\section{Introduction}
It is extremely common to have prior information available on parameter
contraints in statistical models. For example, one may have prior knowledge
that a vector of parameters lies on the probability simplex or satisfies a
particular set of inequality constraints. Other common examples include
shape constraints on functions, positive semidefiniteness of matrices and
orthogonality. There is a very rich literature on optimization subject to
parameter contraints. One common approach is to rely on Lagrange and
Karush-Kuhn-Tucker multipliers \citep{boyd2004convex}. However, simply
producing a point estimate is often insufficient, as uncertainty
quantification (UQ) is a key component of most statistical analyses. Usual
large sample asymptotic theory, for example showing asymptotic normality of
statistical estimators, tends to break down in constrained inference
problems. Instead, limiting distributions may have a complex form that
needs to be rederived for each new type of constraint, and may be
intractable. An appealing alternative is to rely on Bayesian methods for
UQ, including the constraint through a prior distribution having restricted
support, and then applying Markov chain Monte Carlo (MCMC) to avoid the
need for large sample approximations.
Although MCMC is conceptually simple, except for a few limited cases, it is generally difficult to generate random
variable strictly inside constrained space.
 
To overcome this difficulty, one common strategy  is to  reparameterize
 with un-/less constrained parameters at equal or less dimension. The new parameters form functions
that  can always
satisfy the constraint. The transformation, if bijective,  is known as `coordinate system' in manifold embedding literature \citep{nash1954c1,do2016differential}. Examples
include the polar coordinates for data on a hyper-sphere, or stick-breaking construction for Dirichlet distribution on
probability simplex \citep{ishwaran2001gibbs}.  One can then directly assign prior on the less constrained parameters.
Although this strategy has been successful, convenient coordinate system does not always exist; and heavy reparameterzation tends to makes it more
difficult to
induce prior property on the original space. For example, uniformity of unconstrained
parameter in a compact space may not be equivalent to uniformity on the constrained space via transformation. \cite{diaconis2013manifold}
provide a useful tutorial and cautious guide on this subject.

Alternatively, it is typical to rely on customized solution for specific
constraints. One popular strategy is to restrict focus to a prior and
likelihood such that posterior sampling is tractable. For example, for
modeling of data on Stiefel manifolds, von Mises-Fisher and matrix Bingham-von
Mises-Fisher distribution  \citep{khatri1977mises,hoff2009simulation} are
routinely used. Besides limiting consideration to specialized models, another  drawback is that the tractable computation, especially posterior conjugacy, tends to break down under common modeling/data complication, such as matrix symmetry, hierarchical structures, etc.

For these reasons, it is appealing to consider approaches that do not rely on conjugate constrained distributions. Early work \citep{gelfand1992bayesian} suggested using general unconstrained distribution inside a simple truncated space, and running Gibbs sampling ignoring the constraint but only accepting the draws that fall into truncated space. Unfortunately, this method can be highly inefficient if constrained space has a small or zero measure, which
will create a low or zero  acceptance probability. A recent  idea is to run MCMC ignoring the constraint, and then project draws from the unconstrained posterior to the appropriately constrained space. Such an approach was proposed for generalized linear models with order constraints by \cite{dunson2003bayesian}, extended to functional data with monotone or unimodal constraints \citep{gunn2005transformation}, and
recently modified to nonparametric regression with monotonicity
\citep{lin2014monogp} or manifold \citep{lin2016extrinsic} constraints. A
third independent direction utilizes Hamiltonian Monte Carlo (HMC) that incorporates geometric structure with a Riemannian metric \citep{girolami2011riemann}, making proposals strictly inside the constrained space by solving a large
linear system. Although simpler
algorithms using  geodesic flow were proposed for a few selected constrained space \citep{byrne2013geodesic}, compared to the first two strategies that
operates in unconstrained space, 
strictly accommodating the constrained geometry tend to require more customization,
such as computing the metric tensor for different manifolds.

 
The goal of this article is to dramatically expand the families of constrained priors one could use and develop simple computational strategy for more general
constraints. We first introduce a general strategy to adapt common existing distributions into constrained space. To enable simple posterior computation, we {\em relax} the parameter into the neighborhood surrounding the constrained space. This 
approach enjoys the advantages of unconstrained space sampling while approximately
takes into account of the geometry of the constrained space. This relaxation either produces an approximation for posterior under general constraints formed by equality and/or inequality, or an exact solution for several common constrained space such as simplex and Stiefel manifolds. Theoretic
studies are conducted and comparison with existing approaches  are shown in simulations and data
applications.




\section{Constrained Relaxation Methodology}

\subsection{Deriving Constrained Distribution via Conditioning}

Let $\theta\in \mc D$ be a $p$-element parameter of interests. The support $\mc
D$ is a constrained space. The usual Bayesian approach assigns an existing prior
density $\pi_{\mc D}(\theta)$ for $\theta$ only having support $\mc D$, where the available choices are often quite limited. On the other hand, in the un/less-constrained space $\mc R\supset \mc D$, there is a large family of
distributions with well-studied properties. We denote such a distribution by $\pi_{\mc R}(\theta)$. It would be appealing to 
adopt it in the constrained subspace. In this article, we restrict focus on $\mc D$ and $\mc R$ being the subspace of $p$-dimensional Euclidean space, $\mc D \subset\mc R\subseteq \bb R^p$. We use $\mu^p(A)$ to denote the $p$-dimensional Lebesgue measure of set $A$.

Intuitively, one would want to directly apply space truncation to ${\theta\in\mc D}$ on $\pi_{\mc R}(\theta)$, renormalizing by $1/\mu^p(\mc D)$ to obtain the appropriate density. However, this often does not work as many constrained
space has $\mu^p(\mc D)=\int_{\mc D}  \pi_{\mc R}(\theta) \mu^p(d\theta)=0$. Therefore, an alternative strategy is needed.

We now exploit the fact that one can often define a valid conditional distribution,
even conditioned on an event with measure $0$ \citep{kolmogorov1950foundations}.
Starting from  a derived random variable $w=v(\theta)$, as long as $v:\bb R^p\rightarrow \bb R^d$ is  Lipschitz
continuous and measurable with respect to $\pi_{\mc R}$, one can obtain a conditional density given $w=w_0$ 
\be
\label{conditionalDensity}
\pi_{\mc R}(\theta \mid w=w_0) = \frac{1}{m^s(w_0) J(v(\theta))} \pi_{\mc R}(\theta) \mathbbm{1}_{v(\theta)=w_0}
\ee
where $\mathbbm{1}_{E}$ is an indicator equal $1$ when condition
$E$ holds and $0$ otherwise; $J(v(\theta))$ is the Jacobian induced by using
co-area formula \citep{federer2014geometric}, which equals to
$\sqrt{\mbox{det[}D(v(\theta))'
D(v(\theta))]}$ with $D(v(\theta))$ the partial derivative matrix and $\text{det}[.]$
as the determinant; $0<J(v(\theta))<\infty$; $m^s(w_0)=\int_{v(\theta)=w_0} \frac{1}{J(v(\theta))} \pi_{\mc R}(\theta) \mu^s(d\theta) $ is the normalizing constant. Note the $m^s(w_0)$  could be a potentially lower dimensional integral at  $s\le p$, with
$s$ often referred as the `intrinsic' dimension of subspace $\{\theta: v(\theta)=w_0\}$
such that   $0<m^s(w_0)<\infty$.  For example, consider a compact set of $\theta=\{\theta_1,\theta_2,\theta_3\}$ inside a
hyperplane $\{\theta: \theta_1+\theta_2+\theta_3=w_0\}$ in $\bb R^3$: at $p=3$, it has a volume measure of $0$;  but at $s=2$, it has a  positive area measure, which is calculated by integrating over $(\theta_1,\theta_2)$, with $\theta_3=1-\theta_1-\theta_2$ deterministic. Using $m^s(w_0)$
instead of $\mu^p(\mc D)$, \eqref{conditionalDensity} forms a valid conditional density. We provide  a  formal proof and a more rigorous elaboration on  $s$ and $\mu^s(d\theta)$  
in the theory section.

A large family of constraints can be
associated with such function $v(\theta)$ fixed at certain value $w_0$, without
loss of generality, we take $w_0=\bf 0$. We now have
 $\mc D =\{\theta:
v(\theta)={\bf 0}\}$. Although this imposes a
restriction  on the suitable constrained spaces, there is a rich class within this category.
For example, the common  inequality $f(\theta)<0$ can be associated with $v(\theta)=|f(\theta)|_+=\left\{\begin{array}{cc}  0 \text{ if } f(x)\le 0
\\ f(x) \text{ if } f(x)> 0\end{array}\right.$; the 
equality constraint $f(\theta)=0$ can be associated with $v(\theta)=f(\theta)$=0. The former is often used for space truncation, whereas the latter is often used for  embedding manifolds   in $\bb R^p$ \citep{do2016differential}. 

Omitting normalization constant, the derived constrained density is:

\be
\label{constrainedDensity}
\pi_{\mc D}(\theta)=\pi_{\mc R}(\theta\mid v(\theta)={\bf 0})\propto \pi_{\mc R}(\theta) \mathbbm{1}_{\theta\in \mc
D}/J(v(\theta)).
\ee
Obviously, there are potentially more than one suitable $v(\theta)$'s for
this derivation. Often there exists $v(\theta)$ with constant
$J(v(\theta))$ when $\theta\in\mc D$, then the conditional density would appear as if
the result of simple space truncation
from $\mc R$ to $\mc D$.

The derived density \eqref{constrainedDensity} generalizes the family of distributions one can use for constrained space. Since $\pi_{\mc D}(\theta)$
is almost proportional to $\pi_{\mc R}(\theta)$ except for the Jacobian part,
one can often utilize the similar distribution patterns on $\mc D$
and $\mc R$ for guidiing prior selection.


To illustrate, we now derive two distributions on a $(p-1)$-hypersphere, defined as $\mc
D=\{\theta\in
\bb R^p:\theta'\theta=1\}$. The sphere $\mc D$ has intrinsic dimension $(p-1)$ hence measure $0$ in $\bb R^3 $. As the result, a  space truncation and renormalizing would
yield an improper density. Instead, we can define a conditional density using
a simple  function  $v(\theta)=\theta'\theta-1$,
which has $J(v(\theta))=\|2\theta\|=2$ for $\theta\in \mc D$.
 We start with a familiar
location-scale distribution Gaussian distribution with diagonal
covariance $\theta \in \No(F,I\sigma^2)$ as $\pi_{\mc
R}$, where $F\in \mc D $ and $I$ is the identity matrix. Conditioning on $v(\theta)=\theta'\theta-1=0$ yields

\be
\pi_{\mc D}(\theta) &\propto
\exp(-\frac{\|F-\theta\|^2}{2\sigma^2})
\mathbbm{1}_{\theta'\theta=1}/2 \\
& \propto
\exp(\frac{F'}{\sigma^2}\theta)
\mathbbm{1}_{\theta'\theta=1}
,\ee
where the quadratic term $\theta'\theta$ is left out as a constant in the second line. This gives rise to the well-known von
Mises--Fisher distribution \citep{khatri1977mises}. Although the final form
appears more
like an exponential, the behavior of von
Mises--Fisher
on sphere can be explained by its unconstrained parent Gaussian.
In the Gaussian $\pi_{\mc
R}(\theta)$, $\theta$ is symmetrically distributed around $F$, with density
decaying exponentially as $\| \theta-F\|^2$ increases with rate controled by
$({2\sigma^2})^{-1}$; as the constrained
density $\pi_{\mc D}(\theta)$ is proportional $\pi_{\mc R}(\theta)$ on $\mc
D$, it concentrates
similarly since geodesic distance behaves locally similar to Euclidean distance.  Figure~\ref{sphere_examples}(a) shows this distribution parameterized by $F=[1/\sqrt{3},1/\sqrt{3},1/\sqrt{3}]'$ and $\sigma^2=0.1$.

The observation made on von Mises-Fisher immediately suggests we can induce different behavior by using a different unconstrained $\pi_{\mc R}(\theta)$.
% % leo: skip the the multivariate Gaussian example as it doesn't seem original
%  Starting from
% correlated  Gaussian $\pi_{\mc
% R}(\theta)$, $\theta \in \No(F,\Sigma)$, one obtains 
% \be
% \pi_{\mc
% D}(\theta) & \propto\exp(-\frac{1}{2}\theta'\Sigma^{-1}\theta+
% {F'}{\Sigma^{-1}}\theta) \mathbbm{1}_{\theta'\theta=1} \\
% & \propto \exp(\sum_{k=2}^p(-\frac{1}{2}w_k)\theta'\Psi_k'\Psi_k\theta
% + (-\frac{1}{2}w_1)\theta' \Psi_1\Psi'_1\theta+
% + \sum_{k=1}^p w_k F'\Psi_k\Psi'_k\theta ) \mathbbm{1}_{\theta'\theta=1} \\
% & \propto \exp(-\frac{1}{2}\sum_{k=2}^p( w_k - w_1)\theta'\Psi_k'\Psi_k\theta
% + \sum_{k=1}^p w_k F'\Psi_k\Psi'_k\theta )\mathbbm{1}_{\theta'\theta=1}
% \ee
% where $\Psi_k$ and $w_k$ are the $k$th eigenvector and eigenvalue of $\Sigma^{-1}$. This is the $p$-dimensional generalization of the Fisher-Bingham distribution
% proposed by \cite{mardia1975statistics}, who originally focused on $p=3$. Similar to its role in Gaussian covariance, the covariance controls the correlation on the sphere. An illustration with correlated $x$ and $y$ axes is shown in
% Figure~\ref{sphere_examples}(b).
% 
For example, to induce slower decay in density, one can start from a
multivariate $t$-distribution $\pi_{\mc
R}(\theta)$, $t_m(F,I\sigma^2)$ with $m$ degrees of freedom,
mean $F\in \mc D$ and variance $I\sigma^2$, generating a new constrained density 
\be
\pi_{\mc
D}(\theta)
\propto &
(1+\frac{\|F-\theta\|^2}{m\sigma^2})^{-{(m+p)}/{2}}\mathbbm{1}_{\theta'\theta=1}/2\\
\propto &
(1-\frac{F'\theta}{1+m\sigma^2/2})^{-{(m+p)}/{2}}\mathbbm{1}_{\theta'\theta=1}
\ee
As in the $t$-distribution, density decays polynomially as $\|F-\theta\|^2$ increases, at small
$m$, the induced distribution (Figure~\ref{sphere_examples}(b) with $m=3,p=3$) exhibits less concentration than von
Mises--Fisher on the sphere. This can be useful for robust modeling when
there could be `outlier' on the sphere.
We will illustrate more general use by adapting shrinakge prior on similar space later
in the application section. 
\begin{figure}[H]
\begin{subfigure}[b]{0.45\textwidth}
\includegraphics[width=1\textwidth]{sphere_vmf}
\caption{Constrained independent Gaussian distribution}
\end{subfigure}
%\begin{subfigure}[b]{0.30\textwidth}
%\includegraphics[width=1\textwidth]{sphere_fb}
%\caption{Constrained correlated Gaussian distribution}
%\end{subfigure}
\begin{subfigure}[b]{0.45\textwidth}
\includegraphics[width=1\textwidth]{sphere_t}
\caption{Constrained independent $t_3$ distribution}
\end{subfigure}
\caption{Sectional view of random samples from constrained distributions on a unit sphere inside $\bb R^3$. The distributions are derived through conditioning on $\theta'\theta=1$ based on unconstrained densities of (a) $\No( F, \diag\{0.1\})$, 
%(b) $\No(F, \left[\protect\begin{smallmatrix} 0.1 & 0.09 & 0 \\ 0.09 & 0.1 & 0 \\ 0 & 0 &0.1  \protect\end{smallmatrix}\right])$,
 (b) $t_3(F,\diag\{0.1\} )$, where $F=[1/\sqrt{3},1/\sqrt{3},1/\sqrt{3}]'$\\
}
\label{sphere_examples}
\end{figure}



\subsection{ Constraint Relaxation for Posterior Inference}

The conditional distribution \eqref{constrainedDensity} allows one to adapt
simple unconstrained $\pi_{\mc R}(\theta)$ into constrained space. When it
is used as a prior,
with  the likelihood as $ L(y;\theta)$  and $y$ the data, one can obtain posterior:
\be
\label{exactPosterior}
\pi(\theta\mid y) & \propto L(y;\theta)\pi_{\mc D}(\theta) \\
& \propto L(y;\theta) \pi_{\mc
R}(\theta) /J(v(\theta))\mathbbm{1}_{\theta\in \mc D}.
\ee
Clearly, the
posterior also has support only inside $\mc D$ as well due to the inheritance of
$\mathbbm{1}_{\theta\in \mc D}$ from $\pi_{\mc D}(\theta)$. On the hand,
the indicator is often inconvenient for posterior inference. We now develop
 two different strategies to relax the parameter $\theta\in \mc D$ into the neighborhood
of $\mc D$, obtaining posterior $\theta^*$. Then we directly use $\theta^*$ as approximation
or project $\theta^*$ back to $\mc D$ for exact inference. We refer this
strategy as {\bf COnstraint RElaxation (CORE)}.


 \subsubsection{Approximation CORE}

We first present a general approximation strategy. By putting positive mass around $v(\theta)=\bf
0$, and let it decay exponentially, we relax the support of $\theta$ into a
neighborhood surrounding $\mc D $. This generate approximate posterior sample $\theta^*\in \mc R$ via

\be
\label{approximatePosterior}
\tilde{\pi}(\theta^*\mid y)  & = \frac{1}{m(\lambda)} L(y;\theta^*) \pi_{\mc
R}(\theta^*) /J(v(\theta^*)) \exp (- \sum_{k=1}^K |v_k(\theta^*)|^\alpha/\lambda_k),
\ee
where $v_k$ is the $k$th equation in $v(\theta)$, $\alpha$ is
typically chosen as $1$ or $2$ to mimic Laplace or Gaussian kernel, $m(\lambda)$ is a normalizing
constant such that $\int_{\mc R} \tilde{\pi}(\theta^*\mid y) d\theta^*=1$;
 $\lambda_k\ge 0$ is a
tuning parameter controling the concentration around $v(\theta^*)=\bf 0$.
When $\lambda_k=0$ for all $k$, 
\eqref{approximatePosterior} becomes exact; using small $\lambda_k>0$, \eqref{approximatePosterior}
 conventional Monte Carlo approach can be exploited directly in $\mc R$.

We use a toy example with closed-form posterior to illustrate the  approximation. Consider a posterior from a sum-constrained
bivariate Gaussian random vector $[\theta_1,\theta_2]' \mid y \sim \No(0,I)
\mathbbm{1}_{\theta_1+\theta_2-1=0}$. Using
$v(\theta)=\theta_1+\theta_2-1=0$, $J(v(\theta))=\sqrt 2$, the exact posterior \eqref{exactPosterior} is proportional to 
$$
\phi(\theta_1)
\phi(\theta_2)\mathbbm{1}_{v(\theta)=0}
$$
where $\phi(.)$ is the standard
normal density. The exact posterior density has closed-form

$$
\pi(\theta\mid y)=\frac{\sqrt{2}}{\sqrt{2\pi}} \exp(-\frac{(\theta_1-\frac{1}{2})^2}{2/2})
\mathbbm{1}_{\theta_2=1-\theta_1}
$$
corresponding to $\theta_1\mid (\theta_1+ \theta_2=1) \sim
\No(1/2,1/2)$, $\theta_2\mid \theta_1 \sim \delta_{1-\theta_1}(.)$, where
$\delta$ denotes a point mass. Marginally, it is a degenerate Gaussian:
$$\begin{bmatrix} \theta_1 \\ \theta_2 \end{bmatrix} \sim
\No_{\text d} \left(
\begin{bmatrix} \frac{1}{2} \\ \frac{1}{2} \end{bmatrix},
\begin{bmatrix} \frac{1}{2} & -\frac{1}{2}  \\  -\frac{1}{2}  & \frac{1}{2} \end{bmatrix}
\right).$$

Using $ \exp( - (\theta^*_1+\theta^*_2-1)^2/\lambda)$ to
replace $\mathbbm{1}_{v(\theta)=0}$, we obtain approximation $\theta^*_1 \sim \No(\frac{2}{\lambda+4},\frac{\lambda+2}{\lambda+4})$, $\theta^*_2\mid \theta^*_1 \sim \No(\frac{2}{\lambda+2}(1-\theta^*_1),\frac{\lambda}{\lambda+2})$. Marginally, 

$$\begin{bmatrix} \theta^*_1 \\ \theta^*_2 \end{bmatrix} \sim
\No \left(
\begin{bmatrix} \frac{2}{\lambda+4} \\ \frac{2}{\lambda+4} \end{bmatrix},
\begin{bmatrix} \frac{\lambda+2}{\lambda+4} & -\frac{2}{\lambda+4}  \\  -\frac{2}{\lambda+4}  &\frac{\lambda+2}{\lambda+4} \end{bmatrix}
\right).$$
As approximation, one can sample from the non-degenerate distribution using small $\lambda$. Clearly, the approximation error decreases as $\lambda$ gets
smaller, and intuitively becomes exact when $\lambda\rightarrow 0$. We will formalize
this notion in the theory section.

\subsubsection{Data Augmentation CORE}

The previous strategy is generally applicable for approximating the indicator
function. In some less general but
still common cases, one can relax the parameter $\theta$ to $\theta^*$ through some relaxing function
$g$ first, if there is an inverse function $g^{-1}$ to project it back to
$\mc D$, we can use $\theta=g^{-1}(\theta^*)$ to directly reparameterize $\pi_{\mc D}(\theta\mid y)$.


The relaxation $\theta^*=g(\theta;w)$ often requires an auxillary parameter
$w$ that controls the amount of relaxation. Usually $w$ is independent of $\theta$, but can be dependent on $\theta^*$. Treating $w$ as a latent variable from $\pi(w)$  with $\int_{\mc W} \pi(w) dw =1$ and $\mc W$ as
its support, one uses $g^{-1}(\theta^*;w)$ to reparameterize the exact posterior $\pi_{\mc D}(\theta\mid y)$.
Standard variable transformation yields new parameterization


\be
\label{reparameterization}
\pi(\theta^*,w\mid y) & =\pi_{\mc D}(g^{-1}(\theta^*;w)\mid y) \frac{1}{J(g(\theta;w))}  \pi(w) \\
&\propto \frac{\pi_{\mc R}(g^{-1}(\theta^*;w)\mid y) }{ J(v(\theta))\mid_{\theta=g^{-1}(\theta^*;w)} } \frac{1}{J(g(\theta;w))} \pi(w) 
\ee
where   $J(g(\theta;w))$ is the Jacobian of $g$ with respect to $\theta$.
The indicator function $\mathbbm{1}_{\theta\in \mc D}$ disappears since $g^{-1}(\theta^*;w)\in \mc
D$ always holds. It can be verified that transforming $\theta^*=g(\theta;w)$  yields $\pi_{\mc D}(\theta\mid y) \pi(w)$, which is the exact posterior augmented
with latent variable $w$. Therefore, we refer this strategy as data augmentation
constraint relaxation (DA-CORE). Using DA-CORE, one can directly sample $\theta^*$ and $w$ in less constrained
space, and apply $\theta=g^{-1}(\theta^*;w)$ in the end to obtain the $\theta$-marginal.

One simple example of relaxation would be scaling of $\theta$ under  norm constraint. For example, in $(p-1)$-simplex,
$\|\theta\|_1=\sum_{i=1}^p\theta_1=1$ with $\theta\in \mc R=[0,\infty]^p$, one
can augment an independent latent variable $w\in [0,\infty)$ have
$\theta_i^*=\theta_i w$, and the inverse $\theta_i=\theta_i^*/w$ with
$w=\sum_{i=1}^p \theta^*_i$. Suppose exact posterior is a Dirichlet distribution 
$\theta\mid y\sim \text{Dir}(\alpha)$
, 
$$
\pi_{\mc D}(\theta \mid y)\propto \prod_{i=1}^p \theta_i^{\alpha-1}\mathbbm{1}_{\sum_{i=1}^p\theta_i=1},$$
then the relaxed parameterization is $$
\pi(\theta^*,w\mid y) =\pi(\theta^*\mid y)\propto  \prod_{i=1}^{p-1} (\frac{\theta_i^*}{w})^{\alpha-1}  w^{-(p-1)} \pi(w), \quad w=\sum_{i=1}^p \theta^*_i, \quad \theta^*\in(0,\infty)^p.$$

More generally, one can often choose the relaxation $g$ based on the projection $g^{-1}$ that produces $\theta\in \mc D$. To illustrate, we consider the random variable
$\theta$, a $n$-by-$k$ matrix in the Stiefel manifold $\theta\in \mc V(n,k)=\{\theta\in \bb R^{n\times
k}: \theta'\theta=I\}$, where $n\ge k$. The QR-decomposition produces such
an orthonormal matrix, for which a $n$-by-$k$ matrix $X=QR$,
with $Q\in \mc V(n,k)$ and $R$ $k$-by-$k$ upper triangular and diagonal positive;
 the
QR decomposition is unique if $X$ is of rank $k$ \citep{gulliksson1992modifying}, which allows us to
use it as $g^{-1}$. Letting $w$ be a random matrix, $k$-by-$k$, upper triangular and diagonal positive, we have $\theta^*=\theta w$, with $\theta^*$ in rank
$k$. Suppose $\theta$ is $n$-by-$k$ and from the Matrix Bingham--von Mises--Fisher distribution
\citep{hoff2009simulation},
$$\pi_{\mc D}(\theta\mid y)\propto \text{etr}(C'\theta+B\theta'A\theta)  \mathbbm{1}_{\theta'\theta=I_k}$$
where $\text{etr}$ represents the exponential of trace, $A$ is symmetric $n$-by-$n$, $B$ is symmetric $k$-by-$k$ and $C$ is $n$-by-$k$. The relaxed
parameterization is

$$\pi(\theta^*, w\mid y) =\pi(\theta^*\mid y)\propto \text{etr}(C'\theta^*w^{-1}+B(\theta^*w^{-1})'A\theta^*w^{-1} \text{det}^{-1}(w), \quad \text{rank}(\theta^*)=k, \quad w= \text{QR.R}(\theta^*)$$
where $\text{QR.R}$ denotes the function that outputs $R$ matrix in QR decomposition.

In comparison, in other reparameterization such as coordinate system, one can only use equal or less parameters to satisfy the constraint, which can
be  constringent. In DA-CORE, it is generally
more flexible with more parameters, thanks to the data augmentation. Since $w$ is a redundant latent variable, one can
assign $\pi(w)$ to allow greater relaxation, which is commonly associated
with better performance in posterior sampling. More will be illustrated in
the simulation.


 \subsection{Theoretic Properties}

We now present the properties of the proposed approach. We first establish
that the conditioning approach yields valid probability measure.

We focus on $\mc R$ being a $p$-dimensional Euclidean space and the
intrinsic dimension of $\mc D$, $\mbox{dim}(\mc D)=s\le p$  is integer.
Although the
$p$-dimensional Lebesgue measure can have $\mu^p(\mc D)=0$, it is still possible to define a conditional probability given the event ${v(\theta)=\bf 0}$. We utilize the concept of {\it regular
conditional probability} (r.c.p.)
\citep{kolmogorov1950foundations}. For this article to be self-contained,
we list the definition as below (a more complete review can be found in \cite{leao2004regular}).

Let $(X, \mathscr A, \mu)$ be a probability space and $(Y, \mathscr B)$ a measurable space. A function $v$ is measurable if $v:X\rightarrow Y$, $v^{-1}(\mathscr B)\in \mathscr A$.
A r.c.p is a function
$f: Y\times \mathscr A \rightarrow[0,1]$ satisfying:

\begin{enumerate}
        \item $f(y, .)$ is a measure on $(X,\mathscr A)$ for each $y \in
        Y$;
        \item $f(., E)$ is a measurable function on $(Y,\mathscr B)$ for each $E\in \mathscr A$;
        \item For each $E \in \mathscr A$, $F\in \mathscr B$,
        $\mu(E \cap v^{-1}(F))=\int_{F} f(y, E) \mu_y(dy)$, with
        $\mu_y$ the induced measure on $(Y,\mathscr B)$.
\end{enumerate}

Using the previous notation, we write $f(y,E)= P(\theta\in E\mid
v(\theta)=y)=\int_E \pi_{\mc R}(\theta
\mid v(\theta)=y) d\theta$

\begin{remark}
Assuming $J(v(\theta))> 0$ and there is a finite and
non-negative integer $s$ such
that, for some $y\in Y$,
\be
m^{s}(y)=\int \frac{\pi_{\mc
R}(\theta)
\mathbbm{1}_{v({\theta})=y}}{J(v(\theta))}\mu^s(d\theta)\in(0,\infty),
\ee
then
\begin{equation}
P(E\mid v(\theta)=y)=\left\{\begin{array}{ll}  \frac{1}{m_s(y)}\int_{E } \frac{\pi_{\mc
R}(\theta) \mathbbm{1}_{v({\theta})=y}}{J(v(\theta))}\mu^s(d\theta)
& \text{  , if }m_s(y)\in(0,\infty) \\
\delta_{x^*}(E) \text{ with fixed } x^*\in \bb R^p
& \text{  , if }m_s(y)\in\{0,\infty\} \\
\end{array}\right.
\end{equation}
is a valid r.c.p..
\end{remark}
\begin{proof} The first two crieria for r.c.p are trivally satisfied. We
use the Hausdorff measure, the standard tool for geometric measure theory \citep{federer2014geometric}, defined as $\mc H^{s}(A)= \underset{\delta\rightarrow 0}\lim \inf \{ \sum \left[{\text{diam}(S_i)}\right]^s: {A\subseteq \cup S_i, \text{diam}(S_i)\le \delta}, \text{diam}(S_i)=\sup_{x,y\in S}\|x-y\|\}$. We denote the normalized Hausdorff measure as $\bar{\mc H}^{s}(A) =\frac{\Gamma(\frac{1}{2})^{s}}{2^s \Gamma(\frac{s}{2}+1)} \mc H^{s}(A)$. When $s$ is an integer, Lebesgue and normalized Hausdorff measures coincide  $\mu^{s}(A)= \bar{\mc H}^{s}(A)$ \citep{evans2015measure}.

Similar to the proof of (2) of Proposition 2 of \cite{diaconis2013manifold}, using co-area formula \citep{federer2014geometric}:

\be
\mu^{p}(E\cap v^{-1}(F))= & \int \mathbbm{1}_{\theta \in E} \mathbbm{1}_{\theta \in v^{-1}(F)}\pi_{\mc R}(\theta)\mu^p(d\theta) \\
= & \int \left[ \int_{v^{-1}(y)} \mathbbm{1}_{\theta \in E} \mathbbm{1}_{v(\theta) \in F}  \frac{ \pi_{\mc R}(\theta)}{J(v(\theta))} \bar{\mc H}^{s}(d\theta)\right] \mu^{p-s}(dy) \\
= & \int_{F} \left [ \int_{ E}  \frac{ \pi_{\mc R}(\theta) \mathbbm{1}_{v(\theta)=y}}{J(v(\theta))} \mu^{s}(d\theta)  \right] \mu^{p-s}(dy) \\
\ee


For $y\in \{y':m(y')=0\}$, $\int_{ E}  \frac{ \pi_{\mc R}(\theta) \mathbbm{1}_{v(\theta)=y}}{J(v(\theta))} \mu^s(d\theta) \le \int_{ \bb R^{s}}  \frac{ \pi_{\mc R}(\theta) \mathbbm{1}_{v(\theta)=y}}{J(v(\theta))} \mu^s(d\theta) =0$; for $y\in \{y':m(y')=\infty\}$, since $\mu ^{p}(\bb R^p)=\int \mathbbm{1}_{m(y)=\infty}m(y) dy + \int  \mathbbm{1}_{m(y)<\infty}m(y)  dy=1$, one must have $\int \mathbbm{1}_{m(y)=\infty} dy=0$. Combining parts yields
\be
\mu(E\cap v^{-1}(F)) & = \int_{F} \mathbbm{1}_{m(y)\in (0,\infty)}\left[
\int_{E} \frac{\pi_{\mc
R}(\theta) \mathbbm{1}_{v({\theta})=y}}{J(v(\theta))}
\mu^{s}(d\theta) \right] \mu^{p-s}(dy) \\
& = \int_{F} \mathbbm{1}_{m(y)\in (0,\infty)}
\left[ \int_{E}  \frac{1}{m(y)}  \frac{\pi_{\mc
R}(\theta) \mathbbm{1}_{v({\theta})=y}}{J(v(\theta))}
\mu^{s}(d\theta) \right ]  m(y)\mu^{p-s}(dy)\\
& =\int_F f(y,E) \mu_y(dy) 
\ee

\end{proof}

The above remark gives the definition of r.c.p given all $v(\theta)\in Y$. It is possible $m^s(y)=0$ or $\infty$ for some $y\in Y$, but as long as  $m^{s}({\bf 0})\in (0,\infty)$ at certain integer $s$, we would have a valid constrained density 

\be 
\pi_{\mc D}(\theta)=
\frac{1}{m^{s}({\bf 0})}  \frac{\pi_{\mc
R}(\theta) \mathbbm{1}_{v({\theta})={\bf 0}}}{J(v(\theta))}
\ee 

The dimension $s$ is often referred as `intrinsic' dimension of $\mc D$. Formally, one would use a standard concept in geometric measure theory named Hausdorff measure,  \citep{federer2014geometric}. The $d$-dimensional Hausdorff measure is the limit total volume of the $d$-dimenional balls covering $A$, $\mc H^{d}(A)= \underset{\delta\rightarrow 0}\lim \inf \{ \sum \left[{\text{diam}(S_i)}\right]^d: {A\subseteq \cup S_i, \text{diam}(S_i)\le \delta}, \text{diam}(S_i)=\sup_{x,y\in S}\|x-y\|\}$. Then the intrinsic dimension is equal to Hausdorff dimension $s=\inf_{d\ge 0}\{H^d(\mc D)=0\}=\sup_{d\ge 0}\{H^d(\mc D)=\infty\}$, at which the Hausdorff measure transitions from $0$ to $\infty$. Although finding $s$ can be challenging \citep{mardia1975statistics,bowen1979hausdorff}, one could could heuristically test $s\in \{p,p-1,\ldots, 0\}$ if $s$ is known to be an integer. Fortunately, for posterior estimation, there is no need for estimating $s$ or the normalizing constant $m^{s}(\bf 0)$ in Monte Carlo sampling.

We now quantify the approximation error of the approximation CORE \eqref{approximatePosterior}. 
We use $\Pi(.)$ and $\tilde\Pi(.)$ to represent the measures under exact and approximating distributions. 
 For easier notation, we re-parameterize the approximating part as $\exp(-\lambda^{-1}  v(\theta))$ where $\lambda=\max_k \lambda_k$ and $v(\theta)=\sum_{k=1}^d\frac{|v_k(\theta)|^{\alpha}}{\lambda^*_k}$ with $\lambda^*_k=\lambda_k/\lambda$ and define a conditional expectation, $\mathbb{E}(g(\theta) \mid v(\theta)=x)=  \int_{\bb R^s } \frac{ g(\theta) \mathbbm{1}_{v(\theta)=x} \pi_{\mc R}(\theta)}{ J v(\theta) } d \theta.$

 We now assess the behavior of approximation error in terms of 1-Wasserstein distance, as $\lambda$ decreases towards $0$. The 1-Wasserstein distance $W_1(\Pi,\tilde\Pi)$ represents the minimal amount of transport needed to transform one distribution to another. Formally, it is defined as

$$W_1(\Pi,\tilde\Pi)=\underset{\gamma\in \Gamma(\Pi,\tilde\Pi)}{\inf}\int \|\theta-\theta^*\| d\gamma(\theta,\theta^*)$$ 
where $\Gamma(\Pi,\tilde\Pi)$ is the family of all joint measures of with $\Pi$ and $\tilde\Pi$ as the marginals, and $\|\theta-\theta^*\|$ is the
Euclidean distance.



%Due to similar form in prior and posterior, we now introduce some general notation. Let $\pi_{\mc R}(\theta)$ be the normalized density in $\mc R$ such that $\int \pi_{\mc R}(\theta)=1$, which is $\pi_{0,\mc R}(\theta)$ for prior and $L(y;\theta)\pi_{0,\mc R}(\theta)$ for posterior;% Suppose $\Phi:\mathbb R^p \rightarrow \mathbb R^d$ with $p>d$ is Lipschitz, then the co-area formula \citep{federer2014geometric} is,
% \begin{equation}
% \ \int_{\mathbb{R}^n}  f(\theta)J_N\Phi(\theta)\mu^n(d \theta)
% =\ \int_{\mathbb{R}^m}  \int_{\Phi^{-1}(y)}f(\theta) \mc H^{n-m}(d\theta)\mu^m(d y),
% \end{equation}
% where $\mu^k(d\theta)$ a $k$-dimensional Lebesgue measure. 


\begin{remark}
The 1-Wasserstein distance between the measures based on \eqref{exactPosterior} and \eqref{approximatePosterior} has
$$ \underset{\lambda \rightarrow 0}\lim W_1(\Pi,\tilde\Pi)=0.$$
Further, for $\alpha=1$ in \eqref{approximatePosterior},

\begin{equation}
\begin{aligned}
W_1(\Pi,\tilde\Pi) \le \lambda (\frac{k_1 k_2}{m(0)^2} + \frac{k_1}{m(0)}) + \exp(- \lambda^{-1} t )(\frac{k_1}{m(0)^2} + \frac{k_3}{m(0)}),
\end{aligned}
\end{equation}
where $k_1=\underset{g:\|g\|_L\le 1}\sup\underset{t^*\in [0,t)}\sup \|\mathbb{E}(g(\theta^*) \mid v(\theta^*)=t^*)\|$, $k_2= \underset{t^*\in (0,t)}\sup  m(t^{*})$ and $k_3=\underset{g:\|g\|_L\le 1} \sup \mathbb{E}(\| g(\theta )\|)$.
\end{remark}


\begin{proof}[Proof]
Let $g:\mathbb{R}^p\rightarrow \mathbb{R}$ be a 1-Lipschitz continuous function, i.e. $\|g(x)-g(y)\|\le \|x-y\|$, denoted by $\|g\|_L\le 1$. 
By Kantorovich-Rubinstein duality, the 1-Wasserstein distance based on Euclidean metric equals to: 

\begin{equation}
W_1(\Pi,\tilde\Pi)=\underset{g:\|g\|_L\le 1}\sup \int g(x) \Pi(dx) -  \int g(y) \tilde\Pi(dy) 
\end{equation}

Taking $g(\theta)=\exp(-\lambda^{-1}v(\theta))$ in the co-area formula yields
\begin{equation}
\begin{aligned}
m_\lambda
& = \int_\mathbb{R}  \left[ \int_{v^{-1}(x)} \frac{ \exp(- \lambda^{-1} v(\theta) ) \pi_{\mc R}(\theta)}{ J v(\theta) }  \bar{\mc H}^{p-d}(d \theta) \right] \mathbbm{1}_{x \ge 0}  d x \\
& = \int_\mathbb{R}  m(x)\exp(- \lambda^{-1} x ) \mathbbm{1}_{x \ge 0}  d x .
\end{aligned}
\end{equation}

Taking $g(\theta)={\mathbbm{1}_{v(\theta)=0}}$ yields 
\begin{equation}
m_0
= \int_\mathbb{R} \left[ \int_{v^{-1}(y)} \frac{ \pi_{\mc R}(\theta) }{ J v(\theta) }  \bar{\mc H}^{p-d}(d \theta) \right]\mathbbm{1}_{y=0} dy   =  \int_{v^{-1}(0)} \frac{ \pi_{\mc R}(\theta) }{ J v(\theta) } \bar{\mc H}^{p-d}(d \theta) =m(0)
\end{equation}


Clearly $m_\lambda \ge m_0$.

1. Asymptotic result:

We have

\begin{equation}                
\label{wass0}
\begin{aligned}
&\underset{g:\|g\|_L\le 1}\sup\int g(\theta)  \left[ \frac{ \exp(- \lambda^{-1} v(\theta)) } {  m_\lambda}  - 
\frac{ \mathbbm{1}_{v(\theta)=0} } {  m_0} 
\right]  \frac{\pi_{\mc R}(\theta)}{Jv(\theta)}  \bar{\mc H}^{p-d}(d \theta) \\
&= \underset{g:\|g\|_L\le 1}\sup\int_\mathbb{R}  \mathbb{E}(g(\theta) \mid x)  \left[ \frac{ \exp(- \lambda^{-1} x) \mathbbm{1}_{x \ge 0}} {  m_\lambda}  - 
\frac{ \mathbbm{1}_{x=0} } {  m_0} 
\right] d x \\
&=      \underset{g:\|g\|_L\le 1}\sup\int_\mathbb{R}  \mathbb{E}(g(\theta) \mid x)  \left[ \frac{  1} {  m_\lambda}  - 
\frac{ 1 } {  m_0} 
\right]\mathbbm{1}_{x=0} d x  + \underset{g:\|g\|_L\le 1}\sup\int_\mathbb{R}  \mathbb{E}(g(\theta) \mid x)\frac{ \exp(- \lambda^{-1} x)} {  m_\lambda}  
\mathbbm{1}_{x > 0} d x \\
& \le \underset{g:\|g\|_L\le 1}\sup \|\mathbb{E}(g(\theta) \mid 0)\| \left[ \frac{ 1 } {  m_0} -\frac{  1} {  m_\lambda}   
\right] + \frac{1} {  m_0} \underset{g:\|g\|_L\le 1}\sup \int_\mathbb{R}  \|\mathbb{E}(g(\theta) \mid x)\| { \exp(- \lambda^{-1} x)}
\mathbbm{1}_{x > 0} d x \\      
\end{aligned}
\end{equation}


Note $m_\lambda\le \int_\mathbb{R} m(x) \mathbbm{1}_{x \ge 0}  dx =\int_\mathbb{R} \pi_{\mc R}(\theta) =1$. By dominated convergence theorem, 

\begin{equation}
\lim_{\lambda\rightarrow 0}m_\lambda= \int_\mathbb{R}  m(x) \lim_{\lambda\rightarrow 0}\exp(- \lambda^{-1} x ) \mathbbm{1}_{x \ge 0}  d x = m_0.
\end{equation}


Since 
$ \underset{g:\|g\|_L\le 1}\sup \int_\mathbb{R}  \|\mathbb{E}(g(\theta) \mid x)\| { \exp(- \lambda^{-1} x)} dx \le \int_\mathbb{R} \underset{g:\|g\|_L\le 1}\sup \|\mathbb{E}(g(\theta) \mid x)\| { \exp(- \lambda^{-1} x)} dx$,
letting $q_\lambda=       \underset{g:\|g\|_L\le 1}\sup \|\mathbb{E}(g(\theta) \mid x)\| { \exp(- \lambda^{-1} x)}
\mathbbm{1}_{x > 0}  $, we have $0\le q_1-q_{\lambda_1}\le q_1-q_{\lambda_2}$ for $1\ge\lambda_1\ge \lambda_2$, by monotone convergence theorem, $\lim_{\lambda\rightarrow 0}\int [ q_1(x)-q_\lambda(x)]dx = \int [q_1(x)- q_0(x) ]dx$ hence $\lim_{\lambda\rightarrow 0}\int q_\lambda(x)dx =0$. Combining the results yields 



\begin{equation}
\underset{\lambda \rightarrow 0}\lim W_1(\Pi,\tilde\Pi)=0.        \end{equation}

2. Non-asymptotic result:


\begin{equation}
\label{wass1}
\begin{aligned}
\frac{1}{m_0}-\frac{1}{m_\lambda} & \le  \frac{   \int_\mathbb{R}  m(x) \exp(- \lambda^{-1} x ) \mathbbm{1}_{x > 0}  d x } {  m_0^2}  \\
&= \frac{1}{ m_0^2} \left[ \int_0^{t}  m(x) \exp(- \lambda^{-1} x ) dx + \int_t^{\infty}  m(x) \exp(- \lambda^{-1} x ) dx \right] \\
&\le \frac{1}{ m_0^2} \left[  \underset{t^*\in (0,t)}\sup m(t^{*})\int_0^{t} \exp(- \lambda^{-1} x ) dx + \exp(- \lambda^{-1} t )\int_t^{\infty}  m(x) dx  \right] \\
&\le \frac{1}{ m_0^2} \left[\lambda \underset{t^*\in (0,t)}\sup  m(t^{*})  + \exp(- \lambda^{-1} t ) \right] 
\end{aligned}
\end{equation}

\begin{equation}
\label{wass2}
\begin{aligned}
& \underset{g:\|g\|_L\le 1}\sup\int_\mathbb{R}  \|\mathbb{E}(g(\theta) \mid x)\| { \exp(- \lambda^{-1} x)}  \mathbbm{1}_{x > 0} d x \\
&\le \underset{g:\|g\|_L\le 1}\sup\underset{t^*\in (0,t)}\sup \|\mathbb{E}(g(\theta) \mid t^*)\|     
\int_0^{t} \exp(- \lambda^{-1} x ) dx + \exp(- \lambda^{-1} t ) \underset{g:\|g\|_L\le 1}\sup \int_t^{\infty}   \|\mathbb{E}( g(\theta )\mid x)\| dx \\
&\le \underset{g:\|g\|_L\le 1}\sup\underset{t^*\in (0,t)}\sup \|\mathbb{E}(g(\theta) \mid t^*)\|     \lambda + \exp(- \lambda^{-1} t ) \underset{g:\|g\|_L\le 1} \sup \mathbb{E}(\| g(\theta )\|) \\
\end{aligned}
\end{equation}

Combining \eqref{wass0}\eqref{wass1}\eqref{wass2}, $k_1=\underset{g:\|g\|_L\le 1}\sup\underset{t^*\in [0,t)}\sup \|\mathbb{E}(g(\theta) \mid t^*)\|$, $k_2=\underset{g:\|g\|_L\le 1} \sup \mathbb{E}(\| g(\theta )\|)$, $k_3= \underset{t^*\in (0,t)}\sup  m(t^{*})$


\begin{equation}
\begin{aligned}
& \underset{g:\|g\|_L\le 1}\sup \int g(x) \Pi(dx) -  \int g(x) \tilde\Pi(dx) \\
        %&\le  \frac{k_1}{ m_0^2} \left[\lambda k_3  + \exp(- \lambda^{-1} t ) \right] 
        %+ \frac{1} {  m_0} [ k_1       \lambda + \exp(- \lambda^{-1} t )k_2 ]\\     
        & \le \lambda (\frac{k_1 k_3}{m_0^2} + \frac{k_1}{m_0}) + \exp(- \lambda^{-1} t )(\frac{k_1}{m_0^2} + \frac{k_2}{m_0})
        \end{aligned}
        \end{equation}


        \end{proof}

        


The first part shows the asymptotic accuracy of the approximation. The second part shows the rate with non-asymptotic $\lambda$ under mild assumptions. The interpretation for these assumptions is that if in a small space expansion of $\mc D$, defined as $\{\theta^*: v(\theta^*)\in [0,t] \}$, the marginal density of $v(\theta^*)$ and the conditional expectation of Lipschitz functions are bounded $k_1,k_2= \mc O(1)$, and the expected norm of Lipschitz function are smaller than a bound that grows near exponentially $k_3 = \mc O(\lambda \exp(t/\lambda))$, then the distance $W_1(\Pi,\tilde\Pi)$ converges to $0$ in $\mc O(\lambda)$ as $\lambda\rightarrow 0$.



\section{Posterior Computation}

Adapting  unconstrained density into space $\mc D$ often disrupts its posterior conjugacy. Since one can now sample the posterior in $\mc R$ using
 CORE, one can exploit conventional sampling tools such as slice sampling, adaptive Metropolis-Hastings and Hamiltonian Monte Carlo (HMC). In this section, we focus on HMC for its easiness to use and good performance in block updating of parameters.

\subsection{Hamiltonian Monte Carlo under Constraint Relaxation}

We provide a brief overview of HMC for continuous $\theta^*$ under constraint relaxation. Discrete extension is possible via recent work of \cite{nishimura2017discontinuous}. For easy notation, we use $q$ to represent $\theta^*$ under approximation-CORE \eqref{approximatePosterior}, and $\{\theta^*, w\}$ under DA-CORE \eqref{reparameterization}.

In order to sample $q$, HMC introduces an auxillary momentum variable $p \sim \No(0, \mass)$. The covariance matrix $\mass$ is referred to as a \textit{mass matrix} and is typically chosen to be the identity or adapted to approximate the inverse covariance of $q$. HMC then sample from the joint target density $\pi(q, p) = \pi(q) \pi(p) \propto \exp (- H(q, p))$ where, in the case of the posterior under relaxation, 


\begin{equation}
\begin{aligned}
H(q, p)& = U(q)+K(p),\\
\text{where } & U(q) = -\log\pi(q),\\
& K(p) = \frac{p'\mass^{-1} p}{2}.
\end{aligned}
\end{equation}
with $\pi(q)$ is the unnormalized density in \eqref{approximatePosterior} or \eqref{reparameterization}. %$L(\theta^*;y)\pi_{0,\mc R}(\theta^*) \exp( - \sum_k\frac{|v_k(\theta^*)|^\alpha}{\lambda_k}) / J(v(\theta^*))$ based on \eqref{approximatePosterior} and $f(q)= L(g(\theta^*;w);y)\pi_{0,\mc R}(g(\theta^*;w))J(g(\theta^*;w)) / J(v(g(\theta^*;w)))$ based on \eqref{reparametrizedPosterior}. 

From the current state $(q^{(0)},p^{(0)})$, HMC generates a proposal for Metropolis-Hastings algorithm by simulating Hamiltonian dynamics, which is defined by a differential equation:

\begin{equation}
\begin{aligned}
\label{hamiltonian}
\frac{\partial q ^{(t)}}{\partial t} & =\frac{\partial H(q, p)}{\partial p} = \mass^{-1}p,\\
\frac{\partial p^{(t)}}{\partial t}& =-\frac{\partial H(q, p)}{\partial q} = -\frac{\partial U(q)}{\partial q}.
\end{aligned}
\end{equation}

The exact solution to \eqref{hamiltonian} is typically intractable but a valid Metropolis proposal can be generated by numerically approximating \eqref{hamiltonian} with a reversible and volume-preserving  integrator \citep{neal2011mcmc}. The standard choice is the \textit{leapfrog} integrator which approximates the evolution $(q^{(t)},p^{(t)}) \to (q^{(t + \dt)},p^{(t + \dt)})$ through the following update equations:

\begin{equation}
\begin{aligned}
\label{leap-frog}
p \leftarrow p - \frac{\dt}{2} \frac{\partial U}{\partial  q },\quad
q \leftarrow  q  + \dt \mass^{-1}p,\quad
p \leftarrow p -  \frac{\dt}{2}  \frac{\partial U}{\partial  q } 
\end{aligned}
\end{equation}
Taking $L$ leapfrog steps from the current state $(q^{(0)},p^{(0)})$ generates a proposal $(q^{*},p^{*}) \approx (q^{(L \dt)},p^{(L \dt)})$, which is accepted with the probability 
$$1\wedge \exp  \left( - H(q^{*},p^{*}) + H(q^{(0)},p^{(0)}))\right)$$


\subsection{Computing Efficiency and Support Expansion}

Since CORE expands the support from $\mc D$ to $\mc
R$, it is useful to study the effect of space expansion on the computing efficiency. In this section, we provide some  quantification of the effects and provide a practical guidance on choosing $\pi(w)$ or $\lambda$ in the
two strategies.


In understanding the computational efficiency of HMC, it is useful to
consider the number of leapfrog steps to be a function of $\dt$ and
set $L = \lfloor \tau / \dt \rfloor$ for a fixed integration
time $\tau > 0$. In this case, the mixing rate of HMC is completely
determined by $\tau$ in the limit $\dt \to 0$ \citep{betancourt17}. In
practice, while a smaller stepsize $\dt$ leads to a more accurate
numerical approximation of Hamiltonian dynamics and hence a higher
acceptance rate, it takes a larger number of leapfrog steps and
gradient evaluations to achieve good mixing. For an optimal
computational efficiency of HMC, therefore, the stepsize $\dt$ should
be chosen only as small as needed to achieve a reasonable acceptance
rate \citep{beskos13, betancourt14}. A critical factor in determining
a reasonable stepsize is the \textit{stability limit} of the leapfrog
integrator \citep{neal2011mcmc}. When $\dt$ exceeds this limit, the
approximation becomes unstable and the acceptance rate drops
dramatically. Below the stability limit, the acceptance rate $a(\dt)$
of HMC increases to 1 quite rapidly as $\dt \to 0$ and in fact
satisfies $a(\dt) = 1 - \mc O(\dt^4)$ \citep{beskos13}.

For simplicity, the following discussions assume the mass matrix $\mass$ is taken to be the identity. Let $\hess_U(q)$ denote the hessian matrix of $U(q) = - \log \pi(q)$ and let $\xi_1(q)$ denotes the first largest eigenvalue of $\hess_U(q)$. While analyzing stability and accuracy of an integrator is highly problem specific, the linear stability analysis and empirical evidences suggest that, for stable approximation of Hamiltonian dynamics by the leapfrog integrator in $\bb R^p$, the condition $\dt < 2\xi_1(\theta)^{-1/2}$ must hold on most regions of the parameter space \citep{hairer06}.
Besides the eigenvalue, if the support of $q$ is a constrained space $\mc Q$, another limiting factor is roughly the shortest distance to the boundary $\eta (\theta; {\mc Q})= \inf_{q'\not\in \mc Q}\|q'-q\|$. If either $\eta (\theta; {\mc Q})$ or $\xi_1(\theta)^{-1/2}$ 
is close to $0$, the upper bound would be too low to obtain efficient computation.
In constrained model, the parameter space $\mc D$ can have very small $\eta(\theta;\mc D)$. Constraint relaxation can reduce this problem via support
expansion. 

For approximation-CORE \eqref{approximatePosterior}, to control approximation
error, one can choose to relax a subset of constringent constraints. Observing  $\mc D=\cap_{k} D_k$,
each approximation $\exp(-\frac{|v_k(\theta^*)|^\alpha}{\lambda_k})$ corresponds to a constrained space $\mc D_k$.  One practical strategy is that, for $\mc D_k$'s with $\eta(\theta;\mc D_k)\approx 0$, one uses moderate $\lambda_k$ to induce some support expansion (denoted by $\lambda_k\ge \zeta$ with $\zeta$ moderately small but not too close to $0$);  for $\mc D_k$'s without this issue, one uses very small $\lambda_k\approx 0$ to almost always uphold the constraint.
The latter was also suggested by \cite{neal2011mcmc} as creating a high `energy wall'. Noting this could create  inaccuracy of HMC near the boundary with $\lambda_k\approx 0$ under fixed step size,  we use random step size $\epsilon$ at each iteration to reduce the error.


The Hessian $\hess_U(q)$ under approximation-CORE is given by
\begin{equation}
\label{eq:hessian_extrinsic}
\hess_U(q) = -\hess_{\log L(y;\theta^*) \pi_{\mc
R}(\theta^*) /J(v(\theta^*)) }(\theta^*)+\sum_k \lambda_k^{-1} \hess_{|v_k|^\alpha}(\theta^*) \mathbbm{1}_{\theta\not\in \mc D_k},
\end{equation}
where the second term $\lambda_k^{-1} \hess_{v_k}(\theta) \mathbbm{1}_{\theta\not\in \mc D_k}$ is $0$ unless $\theta\not\in \mc D_k$. Since the $\lambda^{-1}_k$'s
in the second term often dominate the eigenvalue, hence $\xi^{-1/2}_1(\theta^*)\approx
\underset{\lambda_k\ge \zeta}{\min}\lambda_k^{1/2}$. A trade-off between approximation accuracy and computational efficiency is involved. Fortunately, as quantified above, the approximation error is often $\mc O( \underset{\lambda_k\ge \zeta}{\max}\lambda_k)
$ and decreases faster than the efficiceny cap $\mc O( \underset{\lambda_k\ge \zeta}{\min}\lambda_k^{1/2})$, as $\lambda_k$ decreases.\ In our experiments, we did find changing from $\lambda_{k}=10^{-4}$ to $10^{-5}$ requires approximately $3$ times of computing budget, due the effect on stability
bound.


On the other hand, since DA-CORE \eqref{reparameterization} does not involve such error trade-off, it is preferred when it is applicable. Letting $\mc Q_\theta\subset \mc D$ denote the support for the constrained $\theta\in \mc D$, the reparameterization changes the support to $Q_{\theta^*}=\{g(\theta;w):\theta\in \mc Q\}$. Therefore, one could choose $\pi(w)$ to substantially increase $\eta(\theta^*; Q_{\theta^*})$.
Since the augmented variable $w$ is redundant, DA-CORE can
be considered as one type of parameter expansion  discovered by \cite{liu1999parameter}, who originally
focused on accelerating  Gibbs sampling of probit regression. 
Although for
greater space expansion,  it is possible to use diffuse or even improper prior for $\pi(w)$ on $Q_{\theta^*}$,
 we recommend assigning $\pi(w)$ loosely centered at $w_0:g(\theta;w_0)=\theta$ (corresponding to when $\theta^*=\theta$), which makes $\theta^*$ a mild relaxation of $\theta$. This ensures no substantial change in $\xi^{-1/2}_1(\theta^*)$ in HMC.

\section{Simulations}
In order to compare against existing approaches on computing efficiency and provide empirical evidence
supporting our previous result, we run simulations on several toy examples
 in this section.
   
\subsection{Gaussian  under Linear Inequality}
We first consider linear models under linear inequality constraints. Although
recent work proposed a new customized prior with posterior conjugacy \citep{danaher2012minkowski},
via our framework, one can simply exploit general Gaussian prior. We sample a bivariate Gaussian $\theta \sim \No \left( \mu, I\sigma^2\right)$ subject to linear inequality $\theta\in(0,1)^2,\theta_1+\theta_2<1$,
which forms a triangle. In two separate settings, choosing $(\mu, \sigma^2)$ as $([0.3,0.3],1/{10})$  induces a wide-spread distribution centered in the interior of $\mc D$, while $([0.7,0.3]',1/10^4)$  induces a  distribution concentrated on the boundary of $\mc D$. As DA-CORE does not appear straightforward
in this case, we use the approximation-CORE  \eqref{approximatePosterior} with $\exp(-\frac{|v(\theta)|}{\lambda})$,  $v(\theta)=|\theta_1+\theta_2-1|_+ + |-\theta_1|_+ + |-\theta_2|_ + + |\theta_1-1|_+ + |\theta_2-1|_+$.  Since the triangle has wide support with $\eta(\theta;\mc D)$ away from $0$, small $\lambda=10^{-8}$
 guarantees almost no approximation error. Figure~\ref{linear_inequality} plots the posterior sample and its contour. Clearly, all posterior fall inside
$\mc D$. To compare, we ran simple rejection sampling with untruncated normal proposal $\No ( \mu, I\sigma^2)$. As expected, it  suffers from a rapidly growing rejection rate from $12\%$ to $51\%$, as $\mu$ moves further away from the center of $\mc D$.

\begin{figure}[H]
\begin{subfigure}[b]{0.45\textwidth}
\includegraphics[width=1\textwidth]{linear_inequal_1}
\caption{Constrained $\No([0.3,0.3],1/{10})$}
\end{subfigure}
\begin{subfigure}[b]{0.45\textwidth}
\includegraphics[width=1\textwidth]{linear_inequal_2}
\caption{Constrained $\No([0.7,0.3],1/{10^4})$}
\end{subfigure}
\caption{Posterior sample of bivariate normal distribution subject to linear inequality constraints $\theta\in(0,1)^2,\theta_1+\theta_2<1$, using HMC with  constraint relaxation. Posterior is spread out around the center (panel (a)) or concentrated on the boundary (panel (b)) of the region.}
\label{linear_inequality}
\end{figure}

\subsection{von Mises--Fisher on Unit Circle}
To illustrate equality constraint relaxation, we generate a simple von Mises--Fisher distribution $\pi_{\mc D}(\theta) \propto \exp(F'\theta)$ on a unit circle $\{(\theta_1,\theta_2):\theta_1^2+\theta_2^2=1\}$. We use $F=(5,5)$ to induce a relatively spread-out  $\theta$ on the manifold.
For sampling,
we compare three strategies: approximate-CORE using $\exp(-\frac{|\theta'\theta -1|}{\lambda})$ for approximating the indicator, DA-CORE
using $\theta_1 = \frac{\theta_1^*}{w}, w= \sqrt{(\theta_1^*)^2+ (\theta_2^*)^2}$ and $\pi(w)\sim \No(1,1)\mathbbm{1}_{w>0}$ and  exact  von Mises--Fisher obtained using `movMF' package.

Unlike the previous linear inequality constraint, the unit circle has narrow
$\eta(\theta;\mc D)=0$ for all $\theta\in \mc D$, therefore, some support expansion is needed
for HMC. We test $\lambda = 10^{-3}$, $10^{-4}$ and $10^{-5}$ for approximation-CORE. To compare the efficiency of HMC, we fix the number of leap-frog steps to $20$ within one iteration HMC, and let software STAN
automatically tune for stable step size. Table~\ref{table_circle} shows the effective sample size per $1000$ iterations, the effective `violation' $|v(\theta)|=|\theta_1+\theta_2-1|$ and the 1-Wasserstein distance $W_1$ as the approximation error. As $W_1$ is numerically computed, to provide a baseline error, we also calculate the average $W_1$ comparing two independent samples from the same exact distribution. The approximation error $W_1$ based on $\lambda= 10^{-5}$ approximation is indistinguishable from this low numerical error, while the other approximations  have slightly larger error but more effective samples. As expected, the DA-CORE is exact
and has high effective sample size. 
   \begin{table}[H]
   \begin{center}
   \tiny
   \begin{tabular}{ c| c | c| c |c | c}
   \hline     
    & \multicolumn{4}{c|}{ HMC based on CORE}     & Exact  \\   
       \hline     
        & \multicolumn{3}{c|}{Approximation}     & DA-CORE  \\           
       \hline           
     &  $\lambda=10^{-3}$ & $\lambda=10^{-4}$ & $\lambda=10^{-5}$ &  
     &  \\
   \hline
   \hline
   $W_1$ & 0.050 & 0.034  & 0.014 & 0.017  & 0.015 \\

   &  (0.019, 0.095) &(0.027, 0.037) &  (0.013,0.025)  & (0.0012,0.026) & (0.0014,0.025)\\

   \hline
   $|v(\theta)| \mid y$ 
   & $9\times 10^{-4} $ 
   & $9\times 10^{-5} $ 
   & $9\times 10^{-6} $ &0 & 0\\
   & $(2.6 \cdot 10^{-5}, 3.3\cdot 10^{-3})$& $(2.0 \cdot 10^{-6}, 3.4\cdot 10^{-4})$& $(2.7 \cdot 10^{-7}, 3.5\cdot 10^{-5})$&  & \\
   \hline
   ESS /1000 Iterations &  751.48  & 260.54 & 57.10 & 788.30    \\
   \hline  
   \end{tabular}
   \end{center}
   \caption{Benchmark of constraint relaxation methods on sampling von--Mises Fisher distribution on a unit circle. For each approximation CORE, average approximation error (with 95\% credible interval, out of $10$ repeated experiments) is computed, and numeric error of $W_1$ is shown under column `exact' as comparing two independent copies from the exact distribution\label{table_circle}.
Effective sample size shows DA-CORE and approximation-CORE with relatively large $\lambda$ have high computing efficiency.
}
   \end{table}

\subsection{Dirichlet on a Simplex}
Lastly, we experiment with a particularly challenging distribution on a 
$(p-1)$-simplex, defined by $\{\theta: \theta\in(0,\infty)^p,\sum_{i=1}^p \theta_i=1\}$. We consider Dirichlet distribution $\text{Dir}(\alpha)$, with
$\pi_{\mc D}(\theta)\propto \prod_{i=1}^p\theta_i^{\alpha-1}$. When the concentration
parameter $\alpha<1$, $\text{Dir}(\alpha)$ exhibits sparse property that some $\theta_i$'s become very close to $0$, which is exploited in topic modeling \citep{wang2009decoupling} and shrinkage
\citep{bhattacharya2015dirichlet} literature. Despite the simple form, the computation can be quite difficult
 if there is large uncerntainty associated with $\theta$ on top of sparsity.
 The distribution will be multi-modal with distribution scattered along the boundary of the simplex (Figure~\ref{simplex}(a)).

To illustrate, we consider $p=3$ and various values of $\alpha\in \{1,0.5,0.1,0.01\}$. We test the performance of approximation-CORE and DA-CORE. To compare, we also test the standard HMC using coordinate
system $\theta_1=\cos^2(\theta_1^*), \theta_2=\sin^2(\theta_1^*)\cos^2(\theta_2^*), \theta_3=\sin^2(\theta_1^*)\sin^2(\theta_2^*)$ for $\theta^*\in (0,2\pi)^2$,
which is equivalent to stick-breaking representation \citep{ishwaran2001gibbs}; and the geodesic
HMC utilizing the geometric flow directly on the simplex \citep{byrne2013geodesic}.
For all HMCs, we fix the number of steps in each iteration to be $30$ and tune the step size to have  effective sample size as large as possible.

Table~\ref{simplex_tb} lists the effective sample sizes under different $\alpha$'s.
As $\alpha$ becomes smaller than $1$, approximation-CORE and geodesic HMC become worse in performance, while DA-CORE
 and coordinate system are much less impacted.
Figure~\ref{simplex}(b) shows at $\alpha=0.01$, the approximation-CORE and geodesic
HMC are stuck for a long time, while DA-CORE works substantially better. As a well-tested reparameterization,  HMC based on coordinate system still works acceptably well in this case.

The difficulty that approximation-CORE encountered was  anticipated. \cite{byrne2013geodesic} have previously reported similar
slow-down of geodesic HMC computing on hyper-Dirichlet  distribution \citep{hankin2010generalization} with $\alpha<1$. Comparing these two approaches, geodesic HMC relies on restricting the kinetic flow on $\mc D$ via its product with the metric tensor, and approximation-CORE relies
on creating high energy wall in the potential energy. The latter can
be viewed as an approximation
to the former, which explains the similarity in performance.


\begin{figure}[H]
\begin{subfigure}[b]{0.45\textwidth}
\includegraphics[width=1\textwidth]{simplex001}
\caption{$2,000$ samples from $\text{Dir}(0.01)$ on $2$-simplex.}
\end{subfigure}
\begin{subfigure}[b]{0.45\textwidth}
\includegraphics[width=1\textwidth]{simplexTrace001}
\caption{Traceplot of $\theta_1$ using 4 types of HMCs.}
\end{subfigure}
\caption{Sampling of Dirichlet on an simplex with distribution concentrated on the boundaries.
Panel(a) illustrates the distribution under $\text{Dir}(0.01)$; Panel(b)
compares the traceplots of 4 different types of HMCs, which are based on: approximation-CORE with $\lambda=10^{-3}$, DA-CORE, geodesic flow on simplex \citep{byrne2013geodesic} and coordinate
system.}
\label{simplex}
\end{figure}


   \begin{table}[H]
     
   \begin{center}
   \tiny
   \begin{tabular}{ l| r | r| r |r | r}
   \hline     
    & \multicolumn{3}{c|}{HMC based on CORE}     & Geodesic HMC  & Coord
    System HMC \\   
        & {Approx $\lambda=10^{-3}$} & {Approx $\lambda=10^{-4}$}      & DA
        &  \\  \hline         
   ESS /1000 Iter. ($\alpha=1$) & 511.43   & 146.07  & 947.53 &  174.14
   & \bf 961.08     \\
      ESS /1000 Iter. ($\alpha=0.5$) & 145.15  & 33.16  &\bf  912.94  & 31.47  & 846.92   \\
            ESS /1000 Iter. ($\alpha=0.1$) &  88.32   & 26.88& \bf 992.75  &28.70  & 875.83    \\
   ESS /1000 Iter. ($\alpha=0.01$) & 20.54 & 3.91 & {\bf 722.44} & 17.26  & 128.55  \\
   \hline  
   \end{tabular}
   \end{center}
   \caption{Average effective sample size per $1000$ iterations in $\text{Dir}(\alpha)$,
   under different $\alpha$. \label{simplex_tb}
}
   \end{table}

   \section{Application: Finding Sparse Basis in a Population of Networks}

 We now consider a real data application in brain network analysis. 
 The brain connectivity structures are obtained in the data set KKI-42 (Landman et al. 2011), which consists of $n=21$ healthy subjects without any history of neurological disease. We take the first scan out of the scan-rescan data
as the input. Each observation is a $V\times V$ symmetric network, recorded as an adjacency matrix $A_i$ for $i=1,\ldots,n$. The regions are constructed via the Desikan et al. (2006) atlas, for a total of V = 68 nodes.
For the $i$th matrix $A_i$, $A_{i,k,l} \in \{0,1\}$ is the element on the $k$th row and $l$th column of $A_i$, with $A_{(i,k,l)}=1$ indicating there is an connection between $k$th and $l$th region, $A_{(i,k,l)}=0$ if there is no connection. The matrix is symmetric due to the undirectedness of the network,
but the diagonal records $A_{(i,k,k)}$ for all $i$ and $k$ are missing due
to the lack of meaning for self-connectivity. 


One scientific interest in neuroscience is to quantify the variation of brain networks and identify the  regions
(nodes) that contribute to it. Extending 
factor analysis to multiple matrices, one appealing approach
is to have the networks share a common factor matrix but let the  loadings
vary across subjects. This
can be considered as a simplified equivalent of three-way
tensor factorization \citep{kolda2009tensor}. Then to selectively identify the important nodes, one natural way is to apply shrinkage on the elements
of factor matrix.


Geometrically, the factor matrix, denoted by $\{U_1,\ldots,U_d\},$ reside on a Stiefel manifold $\mc V(n,d)=\{U: U'U=I_d\}$, where $U=[U_1,\ldots,U_d]$ is the $n\times d$ matrix. Using $r$ to index $1,\ldots,d$, each frame $U_r$ represents a $(n-1)$-hypersphere. Applying shrinkage forces some of its sub-coordinates to be close to $0$, which
is  reducing each $U_r$ onto a lower-dimensional hypersphere.
Although previous work was done using sparse PCA \citep{zou2006sparse} for
continuous outcome,  little work has been done in a probabilistic model for
binary matrices.

To apply shrinkage in the constrained space, we adopt the global-local shrinkage prior  as common in Bayesian  literature (reviewed by
\cite{polson2012local}), which usually takes the form hierarchical structure $\theta_i\mid
\kappa_i  ,\sigma\sim \No(0,\kappa_i \sigma), \quad \kappa_i\sim G_1, \quad \sigma \sim G_2$ with
$\kappa_i$, $\sigma$ as the local and global scale parameters.  However,
when constraining $\theta_i$, one caveat would be only adapting the conditional density $\No(\theta_i;\kappa_i \sigma)$, which  yields intractable normalizing constant involving $\kappa_i \sigma$ in the conditional. This difficulty can be avoided by
reparameterizing\ $\theta_i=\eta_i\kappa_i\sigma$ with $\eta_i\sim\No(0,1)$, and adapting the {\it joint} density of $\{\eta_i,\kappa_i,\sigma\}$ on constrained
space instead. The
joint density will not have intractable constant as long as the hyper-parameters in $G_1$ and $G_2$ are fixed.

We now
take the Dirichlet-Laplace prior \citep{bhattacharya2015dirichlet} as unconstrained distribution
$\pi_{\mc R}$ and adapt it onto Stiefel manifold via \eqref{constrainedDensity}.
   \begin{equation*}
   \begin{aligned}
   & A_{(i,k,l)} \sim \text{Bern}( \frac{1}{1+ \exp(-\psi_{(i,k,l)}- z_{(k,l)})})\\
   & \psi_{(i,k,l)} = \sum_{r=1}^{d}  v_{(i,r)} u_{(k,r)} u_{(l,r)}  \\
   & U'U=I_{d} \text{ with } U=\{u_{(k,r)}\}_{k=1,\ldots,n; r=1,\ldots,d}\\
           & u_{(k,r)}= \eta_{(k,r)}\kappa_{(k,r)}\sigma_{u} \\
   & \eta_{(k,r)}\sim \text{Lap}(0,1), \quad \{\kappa_{(1,r)}\ldots \kappa_{(V,r)}\} \sim \text{Dir}(\alpha),
    \quad \sigma^2_{u}\sim \text{IG}(2,1)\\   
   & z_{(k,l)} \sim \No(0,\sigma^2_z), \quad  \sigma^2_z \sim \text{IG}(2,1)
   \\
   & v_{(i,r)} \sim \No(0,\sigma^2_{v,(r)}), \quad  \sigma^2_{v,(r)} \sim \text{IG}(2,1)
   \end{aligned}
   \end{equation*}
   for $k>l$, $k=2,\ldots, V$, $i=1,\ldots,n$;  $\text{Lap}(0,1)$ denotes
   the Laplace distribution centered at $0$ with scale $1$; $Z=\{z_{(k,l)}\}_{k=1,\ldots,V;l=1,\ldots,V}$ is a symmetric unstructured matrix that serves as the latent mean; $\{ v_{(i,r)}\}_{r=1,\ldots,d}$
is the loading for the $i$th network, with each $v_{(i,r)}>0$; for all other scale parameters $\sigma^2_.$, we
choose weakly informative prior inverse Gamma $\text{IG}(2,1)$, as appropriate for the scale under the logistic link. To induce sparsity in each Dirichlet,
we use $\alpha=0.1$ as suggested by 
\cite{bhattacharya2015dirichlet}.

 
There are two types of constraints in the model,  $U'U=I_d$ and $\sum_{k=1}^V \kappa_{(k,r)}=1$ for $r=1,\ldots,d$. Taking $v_1(U)= U'U-I_d$ and $v_2(\kappa_{(k,r)})=\sum_{k=1}^V \kappa_{(k,r)}-1$ for each $r$, the Jacobian is  constant in \eqref{constrainedDensity}. For posterior computation, we use DA-CORE
as described above.
Using latent variable $w_U$ $d$-by-$d$ upper triangular and positive diagonal
matrix, and $w_{\kappa,{(r)}}>0$  for $r=1,\ldots ,d$, we relax the parameters
to
$$U^*=U w_U, \quad \kappa^*_{(k,r)}=\kappa_{(k,r)} w_{\kappa,{(r)}},$$
which yields re-parameterization via  projection
\begin{equation}
\begin{aligned}
& U=U^* w_U^{-1}, \quad  w_U=\text{QR.R}(U^*), \\
& \kappa_{(k,r)} = \frac{\kappa^*_{(k,r)}}{w_{\kappa,{(r)}}}, \quad  w_{\kappa,{(r)}}= \sum_{k=1}^V \kappa^*_{(k,r)}\\
& \eta_{k,r}= \frac{u_{(k,r)}}{\kappa_{(k,r)}\sigma_{u}},
\end{aligned}   
\end{equation}where $\text{QR.R}$ denotes the function that outputs $\text{R}$ matrix in
QR decomposition.
To control the amount of relaxation, we assign $w_U$ near $I_d$ via $\pi(w_U)\propto \text{etr}\left[ -\frac{(w_U-I_d)'(w_U-I_d)}{\lambda}\right]$ and $w_{\kappa,{(r)}}$
near $1$ via $\pi(w_{\kappa,{(r)}})\propto \exp\left[ -\frac{(w_{\kappa,{(r)}}-1)^2}{\lambda}\right]$ and set $\lambda=10^{-3}$.


For comparison, we test with the specified model (i) against (ii) the  same
model except with simple $u_{(k,r)}\sim \No (0,\sigma^2_u)$ instead of the shrinkage prior and (iii) the  same
model except  without the orthonormality constraint $U'U=I$ and the shrinkage prior.
We run all models for $10,000$ iterations and discard the first $5,000$ iteration
as burn-in.
For each iteration, we run $300$ leap-frog steps. For efficient computing, we truncated $d=20$. 

Table~\ref{network_model} lists the benchmark results. Compared to (i) and
(ii), the unconstrained model (iii) suffers from very low effective sample
size, due to the serious convergence issue in the factor matrix $U$. As explained
by previous findings in matrix/tensor factorization \citep{hoff2016equivariant},   the factor matrix could
scale and rotate without changing the likelihood,  and  substantial improvement
could be obtained by applying orthonormality constraint.


Figure~\ref{network_model_basis}(a)
plots the posterior mean loadings $v_{(i,r)
}$, with each line representing one subject. For all $i=1,\ldots,21$, the
lines drop quickly to near $0$ after $r\ge 5$ in  model (i)
and (ii), but only do so
until $r\ge 10$ in model (iii). This indicates that independent factors are more effective  representation of the span, compared to non-orthogonal ones. Clearly,
(i) shows more variability than (ii) in the loading $v_{(i,r)}$. We validate these models by calculating area under the receiver operating characteristic curve  (AUC) based on the mean predicted probability and the binary outcome $A_{(i,k,l)}$, using the fitted data and the other unused rescan data from the $21$ subjects.
The models (i) and (ii) with orthonomality constraint perform similarly well, and clearly better than 
the unconstrained model (iii) in prediction AUC.  
 \begin{table}[H]
   \begin{center}
   \tiny
   \begin{tabular}{ l| c | c| c }
   \hline     
 Model   &  (i).with shrinkage \&\ orthonormality    & (ii).with  orthonormality only  & (iii).unconstrained \\         
       \hline           
     Fitted AUC &  97.9\%  & 97.1\%  & 96.9\%     \\
   \hline
     Prediction AUC & 96.2\% & 96.2\%  & 93.6\%      \\
   \hline
   ESS /1000 Iterations &  193.72  & 188.10  & 8.15     \\
   \hline  
   \end{tabular}
   \end{center}
   \caption{Comparing 3 models for 21 brain networks
   \label{network_model}}
   \end{table}

Figure~\ref{network_model_basis}(b) compares the models (i) and (ii) over the top $6$ frames of $U_r$, with  $r$
 re-ordered such that $\sigma^2_{v,(1)}\ge \sigma^2_{v,(2)}\ge \ldots \ge \sigma^2_{v,(d)}$. The posterior of $U_1,U_2,U_3$ look very similar between the two, whereas $U_4,U_5,U_6$ have a considerable subset
of points close to 0 in the model with shrinkage prior.

\begin{figure}[H]
\centering
\begin{subfigure}[b]{0.8\textwidth}
\includegraphics[width=1\textwidth]{network_loading}
\caption{Posterior mean of the loadings $v_{i,r}$ for 21 subjects using three
models. Each line represents the loadings for one subject over $r=1,\ldots10$.}
\end{subfigure}
\begin{subfigure}[b]{1\textwidth}
\includegraphics[width=1\textwidth]{network_factor.pdf}
\caption{Posterior mean and pointwise $95\%$ credible interval of  the factors
$U_1,\ldots,
U_6$ in the two constrained models. }
\end{subfigure}
\caption{Loadings and factors estimates of the network models. Panel (a) compares
the varying loadings of the subjects in three models; Panel (b) compares
the estimated shared factors with and without the shrinkage prior (model (iii) is omitted due to non-convergence in the factors). \label{network_model_basis}}
\end{figure}




\section{Discussion}

Parameter constraint often limits the flexibility to develop new model and creates huge burden in developing efficient posterior sampling algorithms. In this article, we develop a formal strategy to utilize the large pool of distributions in the constrained space, and propose a constraint relaxation approach to allow simple implementation for posterior estimation. For common constrained space that can be projected to via a function, we propose an exact algorithm based on data augmentation; for more general problem, we propose an approximation approach.
This strategy works well for general equality and inequality constraints.


The future work of this research may include tackling the  `doubly intractable' problem. This issue is common when the data is on the constrained space, or the constrained prior has hyper-parameters to estimate. In the data application, we show that a reparameterization strategy works for some shrinkage priors, but clearly, 
more general treatment is needed. We expect our work to be compatible to the existing solutions \citep{murray2012mcmc,rao2016data, stoehr2017noisy}. 

\bibliography{reference}
\bibliographystyle{chicago}
\end{document}


