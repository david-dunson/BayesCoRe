\documentclass[10pt]{article}

\addtolength{\oddsidemargin}{-.875in}
\addtolength{\evensidemargin}{-.875in}
\addtolength{\textwidth}{1.75in}

\addtolength{\topmargin}{-.875in}
\addtolength{\textheight}{1.75in}

\openup 1em

%macro for commenting
\usepackage{color}
\newcommand{\leo}[1]{{\color{blue}{LD: #1}}}

% \newcommand{\Xbeta}{ X_i \theta}
\newcommand{\xbeta}{ x_i \beta}
\newcommand{\xtheta}{ x_i \theta}
% \newcommand{\xbetaij}{ x_{ij}^T \theta}
\newcommand{\sgamma}{s_{ij}^T\gamma_i}

\usepackage[round]{natbib}

\usepackage{rotating}
\usepackage{graphicx}
\usepackage{subcaption}

\usepackage{float}
\usepackage{bbm}

\usepackage{amsthm,amsmath, amssymb} 
\usepackage{mathrsfs}
\usepackage{subcaption}
\usepackage{nicefrac}

\newtheorem{theorem}{Theorem}
\newtheorem{lemma}{Lemma}
\newtheorem{corollary}{Corollary}
\newtheorem{remark}{Remark}


\usepackage{algorithm}
\usepackage{algpseudocode}

%\usepackage{mhequ}
\newcommand{\be}{\begin{equation}\begin{aligned}}
\newcommand{\ee}{\end{aligned}\end{equation}}
\newcommand{\bb}[1]{\mathbb{#1}}
\newcommand{\mc}[1]{\mathcal{#1}}
\DeclareMathOperator{\Binom}{Binomial}
\DeclareMathOperator{\No}{No}
\DeclareMathOperator{\PG}{PG}
\DeclareMathOperator{\IG}{Inverse-Gamma}
\DeclareMathOperator{\Ga}{Gamma}
\DeclareMathOperator{\Bern}{Bernoulli}
\DeclareMathOperator{\U}{Uniform}
\DeclareMathOperator{\Poi}{Poisson}
\DeclareMathOperator{\NB}{NB}
\DeclareMathOperator{\cov}{cov}
\DeclareMathOperator{\var}{var}
\DeclareMathOperator{\diag}{diag}
\DeclareMathOperator{\Diag}{Diag}
\newcommand{\KL}[2]{\textnormal{KL}\left(#1 \parallel #2\right)}

\DeclareMathOperator{\1}{\mathbbm{1}}


\DeclareMathOperator{\bigO}{\mc O}



\thispagestyle{empty}
\baselineskip=28pt

\title{\textbf{Notes on distance between intrinsic and extrinsic priors}}
\author{}
\date{}
\begin{document}
\maketitle

\subsection{Property of Extrinsic Prior}

We now study the properties of the extrinsic prior. One important task is to quantify the difference between extrinsic and intrinsic priors. We first focus on the first case when $\int_{\mc D} \pi_{0,\mc R}(\theta)d\theta>0$.

%distance between extrinsic and intrinsic
\begin{remark}
Let $M_1= \int_{\mc D} \pi_{0,\mc R}(\theta)d\theta$ and $M_2 = \int_{\mc R} \pi_{0,\mc R}(\theta) \mc K(\theta;\mc D)d\theta$, when $M_1>0$, the total variation distance between the measures of extrinsic and intrinsic prior
$$||\pi_{0,\mc D}(\theta), \tilde{\pi}_{0,\mc D}(\theta) ||_{TV} = 1 - \frac{M_1}{M_2} \le \frac{\int_{\theta  \in \mc R \setminus \mc D} \pi_{0,\mc R}(\theta) \mc K(\theta;\mc D)d\theta}{M_1}$$.
\end{remark}
proof:
{via definition of total variation distance and $K(\theta;\mc D)=1$ when $\theta\in\mc D$, $0$ otherwise.}


In the case of exponential smoothing function, we have:
\begin{corollary}
Let $M_1= \int_{\mc D} \pi_{0,\mc R}(\theta)d\theta>0$ and $\mc K(\theta; D) = \prod_{k=1}^m \exp( -v_k(\theta)/\lambda_k)$, one sufficient condition to have
$$\lim_{\text{ all } \lambda_k\rightarrow 0}||\pi_{0,\mc D}(\theta), \tilde{\pi}_{0,\mc D}(\theta) ||_{TV} = 0$$
is that $\pi_{0,\mc R}(\theta)$ is proper, $\int_{\mc R} \pi_{0,\mc R}(\theta) d\theta<\infty$.
\end{corollary}
proof:
{via dominated convergence theorem}

Rewriting $\mc K(\theta; D) = \exp(-v(\theta)/\lambda)$ with $\lambda = \sup_k \lambda_k$, $v(\theta)=\lambda\sum_{k=1}^m\frac{ v_k(\theta)}{\lambda_k}$, we obtain the convergence rate:

\begin{remark}
Assuming $M_3= \int_{\mc R \setminus \mc D} \pi_{0,\mc R}(\theta) d\theta<\infty$, and $f(v)$ be the density of $v(\theta)$ as the transform of $\pi_{0,\mc R}(\theta)/M_3$. If there exists $t>0$ such that $f(v) < \infty$ for $v<t$,
$$\int_0^\infty {\pi_{0,\mc R}(\theta)}exp(- \frac{v(\theta)}{\lambda}) d \theta \le 
2 {M_3} \exp(-\frac{t}{\lambda}) + {M_3} \sup_{t^*\in(0,t)} {f(t^*)}\lambda 
$$
\end{remark}
proof:

\begin{equation}
\begin{aligned}
\int_0^\infty {f(v)} \exp(- \frac{v}{\lambda}) d v
= & \int_0^t {f(v)} \exp(- \frac{v}{\lambda}) d v + \int_t^\infty {f(v)} \exp(- \frac{v}{\lambda}) d v \\
% \le &  t\frac{f(t^*)}{M_3} \exp(- \frac{t^*}{\lambda})+ \exp(- \frac{t}{\lambda})  \int_t^\infty \frac{f(v)}{M_3} d v \\
% \le &  t\frac{f(t^*)}{M_3} \exp(- \frac{t^*}{\lambda})+ \exp(- \frac{t}{\lambda})\\
\le & {F(t)} \exp(-\frac{t}{\lambda}) + 
\frac{1}{\lambda}\int_0^t {F(v)} \exp(-\frac{v}{\lambda})dv + \exp(-\frac{t}{\lambda}) \\
= & ({F(t)} +1) \exp(-\frac{t}{\lambda}) + 
\frac{1}{\lambda}\int_0^t {f(v^*)} v\exp(-\frac{v}{\lambda})dv \\
\le & ({F(t)} +1) \exp(-\frac{t}{\lambda}) + \sup_{t^*\in(0,t)} {f(t^*)}
\int_0^t  \frac{1}{\lambda}v\exp(-\frac{v}{\lambda})dv \\
\le & 2 \exp(-\frac{t}{\lambda}) + \sup_{t^*\in(0,t)} {f(t^*)}\lambda 
\end{aligned}
\end{equation}
where $F(v)=\int_0^v f(x)dx$ and the third step is based on mean value theorem with $v^*\in (0,v)$. Rearranging term yields the result.  $\blacksquare$

That is, for $\lambda$ small, the total varation distance converge to $0$ in $O(\lambda)$. 

We now examine the second case when ${ \int_{\mc D} \pi_{0,\mc R}(\theta)d\theta }=0$.


%distance between extrinsic and intrinsic
\begin{remark}
Let $M_1(\mc D^+)= \int_{\mc D^+} \pi_{0,\mc R}(\theta)d\theta$ and $M_2 = \int_{\mc R} \pi_{0,\mc R}(\theta) \mc K(\theta;\mc D)d\theta$, with $\mc D^+$ chosen such that $M_1(\mc D^+)>0$ and $M_1(\mc D^+)<M_2$, the total variation distance between the measures of extrinsic and intrinsic prior
\begin{equation}
\begin{aligned}
||\pi_{0,\mc D}(\theta), \tilde{\pi}_{0,\mc D}      (\theta) ||_{TV} = &1 - \frac{\lim_{\mc D^+\supset \mc D}\int_{\theta  \in \mc D^+} \pi_{0,\mc R}(\theta) \mc K(\theta;\mc D)d\theta}{M_2} \\
= &\frac{\lim_{\mc D^+\supset \mc D}\int_{\theta  \in \mc R \setminus \mc D^+} \pi_{0,\mc R}(\theta) \mc K(\theta;\mc D)d\theta}{M_2}
	\end{aligned}
\end{equation}
\end{remark}


\bibliography{reference}
\bibliographystyle{chicago}

\end{document}
