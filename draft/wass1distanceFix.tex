%&latex
\documentclass[10pt,fleqn]{article}
\pdfoutput=1

\addtolength{\oddsidemargin}{-.875in}
\addtolength{\evensidemargin}{-.875in}
\addtolength{\textwidth}{1.75in}

\addtolength{\topmargin}{-.875in}
\addtolength{\textheight}{1.75in}

\openup 1em

%macro for commenting
\usepackage{color}
\newcommand{\leo}[1]{{\color{blue}{Leo: #1}}}

% \newcommand{\Xbeta}{ X_i \theta}
\newcommand{\xbeta}{ x_i \beta}
\newcommand{\xtheta}{ x_i \theta}
% \newcommand{\xbetaij}{ x_{ij}^T \theta}
\newcommand{\sgamma}{s_{ij}^T\gamma_i}

\usepackage[round]{natbib}

\usepackage{rotating}
\usepackage{graphicx}
\usepackage{subcaption}

\usepackage{float}
\usepackage{bbm}

\usepackage{amsthm,amsmath, amssymb} 
\usepackage{mathrsfs}
\usepackage{subcaption}
\usepackage{nicefrac}

\usepackage{xcolor}
\newcommand{\aki}[1]{\textcolor{red}{Aki: #1}}

\newtheorem{theorem}{Theorem}
\newtheorem{lemma}{Lemma}
\newtheorem{corollary}{Corollary}
\newtheorem{remark}{Remark}
\newtheorem{example}{Example}


\usepackage{algorithm}
\usepackage{algpseudocode}

%\usepackage{mhequ}
\newcommand{\be}{\begin{equation}\begin{aligned}}
\newcommand{\ee}{\end{aligned}\end{equation}}
\newcommand{\bb}[1]{\mathbb{#1}}
\newcommand{\mc}[1]{\mathcal{#1}}
\DeclareMathOperator{\Binom}{Binomial}
\DeclareMathOperator{\No}{No}
\DeclareMathOperator{\PG}{PG}
\DeclareMathOperator{\IG}{Inverse-Gamma}
\DeclareMathOperator{\Ga}{Gamma}
\DeclareMathOperator{\Bern}{Bernoulli}
\DeclareMathOperator{\U}{Uniform}
\DeclareMathOperator{\Poi}{Poisson}
\DeclareMathOperator{\NB}{NB}
\DeclareMathOperator{\cov}{cov}
\DeclareMathOperator{\var}{var}
\DeclareMathOperator{\diag}{diag}
\DeclareMathOperator{\Diag}{Diag}
\newcommand{\KL}[2]{\textnormal{KL}\left(#1 \parallel #2\right)}
\DeclareMathOperator{\1}{\mathbbm{1}}
\DeclareMathOperator{\bigO}{\mc O}
\newcommand{\dt}{\epsilon} % Stepsize of leapfrog
\newcommand{\mass}{M} % Mass matrix
\newcommand{\hess}{\mathbf{H}} % Hessian notation.



\thispagestyle{empty}
\baselineskip=28pt

\title{\textbf{Constraint Relaxation for Bayesian Modeling with Parameter Constraints}}
\author{Leo Duan,  Alex Young, Akihiko Nishimura, David Dunson}
\date{}
\begin{document}

\maketitle
%\clearpage\pagebreak\newpage
\pagenumbering{arabic}

Assume the approximating function for $\{v_{j}(\theta)=0\}_{j=1}^s$ take the form
$\exp(-\frac{ \|v(\theta)\|_1}{\lambda})=\exp(-\frac{ \sum_j|v_j(\theta)|}{\lambda})$, with $\lambda >0$ a scalar. And
$\mc R$ and $\{v_j(\theta)\}_{j=1}^s$ satisfy $\int \mathbbm{1}_{v(\theta)= x } \pi_{\mc R}(\theta)  \bar{\mc H}^{p-s}(d\theta) \in (0,\infty)$  for any $x\in \mathcal
X =\{v(\theta);\theta\in \mc R\}$.
We denote conditional expectation $\mathbb{E}(g(\theta) \mid v(\theta)=x)=\mathbb{E}(g(\theta) \mid  x)=\int_{v^{-1}(x)} g(\theta)\frac{\pi_{\mc R}(\theta)}{Jv(\theta)}  \bar{\mc H}^{p-s}(d \theta)$. 
\begin{remark}
The 1-Wasserstein distance between the measures based on \eqref{exactPosterior} and \eqref{approximatePosterior} has
$$ \underset{\lambda \rightarrow 0}\lim W_1(\Pi,\tilde\Pi)=0.$$
Further, for $\alpha=1$ in \eqref{approximatePosterior},

\begin{equation}
\begin{aligned}
W_1(\Pi,\tilde\Pi) \le  (2\lambda)^{s} (\frac{k_1 k_3}{m_0^2} + \frac{k_1}{m_0}) + \exp(- \lambda^{-1} t )(\frac{k_1}{m_0^2} + \frac{k_2}{m_0}),
\end{aligned}
\end{equation}
where $k_1= \underset{g:\|g\|_L\le 1}\sup \quad \underset{x^{*}:\|x^{*}\|_1\in(0,t]\cap
\mathcal X}\sup\|\mathbb{E}(g(\theta) \mid x^*)\|$, $k_2=\underset{g:\|g\|_L\le 1} \sup \mathbb{E}(\| g(\theta )\| \mathbbm{1}_{v(\theta)\in
\mathcal X}) $, $k_3= \underset{x^{*}:\|x^{*}\|_1\in(0,t]\cap
\mathcal X}\sup m(x^{*})$.
\end{remark}


\begin{proof}[Proof]

The co-area formula from \cite{federer2014geometric} is,
 \begin{equation}
 \ \int_{\mathbb{R}^p}  f(\theta)J\Phi(\theta)\mu^p(d \theta)
 =\ \int_{\mathbb{R}^{s}} \left[   \int_{\Phi^{-1}(x)}f(\theta)  \bar{\mc H}^{p-s}(d\theta) \right]\mu^{s}(d x),
 \end{equation}
 where $\mu^k(d\theta)$ a $k$-dimensional Lebesgue measure and $\bar{\mc H}^{s}(d\theta)$ is a $s$-dimensional normalized Hausdorff measure.


Let $g:\mathbb{R}^p\rightarrow \mathbb{R}$ be a 1-Lipschitz continuous function, i.e. $\|g(x)-g(y)\|\le \|x-y\|$, denoted by $\|g\|_L\le 1$. 
By Kantorovich-Rubinstein duality, the 1-Wasserstein distance based on Euclidean metric equals to: 

\begin{equation}
W_1(\Pi,\tilde\Pi)=\underset{g:\|g\|_L\le 1}\sup \int g(x) \Pi(dx) -  \int g(y) \tilde\Pi(dy) 
\end{equation}

By assumption, $\pi_{\mc R}(\theta)= \pi_{\mc R}(\theta) \mathbbm{1}_{v(\theta)\in
\mathcal X}$. Taking $f(\theta)=\frac{ \exp(- \lambda^{-1}\| v(\theta) \|_1) \pi_{\mc R}(\theta) \mathbbm{1}_{v(\theta)\in
\mathcal X}}{ J v(\theta) }$ and $\Phi(\theta)=v(\theta)$ in the co-area formula yields
\begin{equation}
\begin{aligned}
m_\lambda
& = \int_{\mathbb{R}^{s}}\left[ \int_{v^{-1}(x)} \frac{ \exp(- \lambda^{-1}\| v(\theta) \|_1) \pi_{\mc R}(\theta)}{ J v(\theta) }  \bar{\mc H}^{p-s}(d \theta) \right]  \mathbbm{1}_{x\in
\mathcal X}\mu^{s}(d x) \\
& = \int_{\mathbb{R}^{s}}\left[ \int_{v^{-1}(x)} \frac{  \pi_{\mc R}(\theta)}{ J v(\theta) }  \bar{\mc H}^{p-s}(d \theta) \right] \exp(- \lambda^{-1}\| x \|_1) \mathbbm{1}_{x\in
\mathcal X}\mu^{s}(d x) \\
& = \int_{\mathbb{R}^{s}}  m(x)\exp(- \lambda^{-1} \| x\|_1 )  \mathbbm{1}_{x\in
\mathcal X}\mu^{s}(d x)  .
\end{aligned}
\end{equation}

Taking $f(\theta)=\frac{\pi_{\mc R}(\theta)}{ J v(\theta) } \mathbbm{1}_{v(\theta)=
\bf 0}$ and $\Phi(\theta)=v(\theta)$ yields 


\begin{equation}
m_0
= \int_{\mathbb{R}^{s}} \left[ \int_{v^{-1}(x)} \frac{ \pi_{\mc R}(\theta) }{ J v(\theta) }  \bar{\mc H}^{p-
s}(d \theta) \right]\mathbbm{1}_{x={\bf 0}} \mu^{s}(d x)    =  \int_{v^{-1}({\bf 0})} \frac{ \pi_{\mc R}(\theta) }{ J v(\theta) } \bar{\mc H}^{s}(d \theta) =m({\bf 0})
\end{equation}


Clearly $m_\lambda \ge m_0$.

1. Asymptotic result:



\begin{equation}                
\label{wass0}
\begin{aligned}
&\underset{g:\|g\|_L\le 1}\sup\int_{\mathbb{R}^{s}}\int_{v^{-1}(x)} g(\theta)  \left[ \frac{ \exp(- \lambda^{-1} \|v(\theta)\|_1) } {  m_\lambda}  - 
\frac{ \mathbbm{1}_{v(\theta)= {\bf 0}} } {  m_0} 
\right]  \frac{\pi_{\mc R}(\theta)}{Jv(\theta)}  \bar{\mc H}^{p-s}(d \theta)\mathbbm{1}_{x\in
\mathcal X}
   \mu^{s}(d x) \\
&= \underset{g:\|g\|_L\le 1}\sup\int_{\mathbb{R}^{s}}  \mathbb{E}(g(\theta) \mid  x)  \left[ \frac{ \exp(- \lambda^{-1} \|x\|_1) } {  m_\lambda}  - 
\frac{ \mathbbm{1}_{x=\bf 0} } {  m_0} 
\right] \mathbbm{1}_{x\in
\mathcal X}\mu^{s}(d x) \\
&=      \underset{g:\|g\|_L\le 1}\sup\int_{\mathbb{R}^{s}}  \mathbb{E}(g(\theta) \mid x)  \left[ \frac{  1} {  m_\lambda}  - 
\frac{ 1 } {  m_0} 
\right]\mathbbm{1}_{x= \bf 0} \mu^{s}(d x)  + \underset{g:\|g\|_L\le 1}\sup\int_{\mathbb{R}^{s}}  \mathbb{E}(g(\theta) \mid x)\frac{ \exp(- \lambda^{-1} \| x\|_1)} {  m_\lambda}  
\mathbbm{1}_{x \neq \bf 0} \mathbbm{1}_{x\in
\mathcal X}\mu^{s}(d x) \\
& \le \underset{g:\|g\|_L\le 1}\sup \|\mathbb{E}(g(\theta) \mid {\bf
0})\| \left[ \frac{ 1 } {  m_0} -\frac{  1} {  m_\lambda}   
\right] + \frac{1} {  m_0} \underset{g:\|g\|_L\le 1}\sup\int_{\mathbb{R}^{s}}  \mathbb{\| E}(g(\theta) \mid x)\|{ \exp(- \lambda^{-1} \| x\|_1)}   
\mathbbm{1}_{x \neq \bf 0} \mathbbm{1}_{x\in
\mathcal X}\mu^{s}(d x) \\      
\end{aligned}
\end{equation}


Note $m_\lambda\le  \int_{\mathbb{R}^{s}}  m(x)\mathbbm{1}_{x\in
\mathcal X}\mu^{s}(d x) =\int_{\mathbb{R}^{p}} \pi_{\mc R}(\theta)d \theta =1$. By dominated convergence theorem, 

\begin{equation}
\lim_{\lambda\rightarrow 0}m_\lambda=  \int_{\mathbb{R}^{s}}  m(x)\lim_{\lambda\rightarrow 0}\exp(- \lambda^{-1} \| x\|_1 )  \mathbbm{1}_{x\in
\mathcal X}\mu^{s}(d x)   = m_0.
\end{equation}


Since 
$$ \underset{g:\|g\|_L\le 1}\sup\int_{\mathbb{R}^{s}} \| \mathbb{E}(g(\theta) \mid x) \|{ \exp(- \lambda^{-1} \| x\|_1)}   
\mathbbm{1}_{x \neq \bf 0} \mathbbm{1}_{x\in
\mathcal X}\mu^{s}(d x) \le\int_{\mathbb{R}^{s}}  \underset{g:\|g\|_L\le 1}\sup \mathbb{\|E}(g(\theta) \mid x)\|{ \exp(- \lambda^{-1} \| x\|_1)}   
\mathbbm{1}_{x \neq \bf 0} \mathbbm{1}_{x\in
\mathcal X}\mu^{s}(d x),$$
letting $q_\lambda=       \underset{g:\|g\|_L\le 1}\sup \mathbb{\|E}(g(\theta) \mid x)\|{ \exp(- \lambda^{-1} \| x\|_1)}   
\mathbbm{1}_{x \neq \bf 0}  \mathbbm{1}_{x\in
\mathcal X}$, for fixed $x$, we have $0\le q_1-q_{\lambda_1}\le q_1-q_{\lambda_2}$ for any two numbers in the series $1\ge\lambda_1\ge \lambda_2$, by monotone convergence theorem, $\lim_{\lambda\rightarrow 0}\int [ q_1(x)-q_\lambda(x)]dx = \int [q_1(x)- q_0(x) ]dx$ hence $\lim_{\lambda\rightarrow 0}\int q_\lambda(x)dx =0$. Combining the results yields 



\begin{equation}
\underset{\lambda \rightarrow 0}\lim W_1(\Pi,\tilde\Pi)=0.        \end{equation}

2. Non-asymptotic result:


\begin{equation}
\label{wass1}
\begin{aligned}
\frac{1}{m_0}-\frac{1}{m_\lambda} & \le  \frac{    \int_{\mathbb{R}^{s}}  m(x)\exp(- \lambda^{-1} \| x\|_1 )\mathbbm{1}_{x \neq \bf 0}\mathbbm{1}_{x\in
\mathcal X}\mu^{s}(d x) } {  m_0^2}  \\
&= \frac{1}{ m_0^2} \left[ \int_{\mathbb{R}^{s}} \mathbbm{1}_{\|x\|_1\in(0,t]}  m(x) \exp(- \lambda^{-1} \| x\|_1  )\mathbbm{1}_{x\in
\mathcal X}\mu^{s}(d x) + \int_{\mathbb{R}^{s}}
 \mathbbm{1}_{\|x\|_1\in(t,\infty)}  m(x) \exp(- \lambda^{-1} \| x\|_1  )
 \mathbbm{1}_{x\in
\mathcal X} \mu^{s}(d x) \right] \\
&\le \frac{1}{ m_0^2} \bigg[  \underset{x^{*}:\|x^{*}\|_1\in(0,t]\cap
\mathcal X}\sup m(x^{*}) \int_{\mathbb{R}^{s}}\mathbbm{1}_{\|x\|_1\in(0,t]}  \exp(- \lambda^{-1} \|x\|_1 )\mathbbm{1}_{x\in
\mathcal X}\mu^{s}(d x) \\ &+ \exp(- \lambda^{-1} t) \int_{\mathbb{R}^{s}}\mathbbm{1}_{\|x\|_1\in(t,\infty]}  m(x)\mathbbm{1}_{x\in
\mathcal X}\mu^{s}(d x)   \bigg] \\
&\le \frac{1}{ m_0^2} \left[(2\lambda)^s  \underset{x^{*}:\|x^{*}\|_1\in(0,t]\cap
\mathcal X}\sup m(x^{*})  + \exp(- \lambda^{-1} t ) \right] 
\end{aligned}
\end{equation}
where 
\be
\int_{\mathbb{R}^{s}}\mathbbm{1}_{\|x\|_1\in(0,t]}  \exp(- \lambda^{-1} \|x\|_1 )\mathbbm{1}_{x\in
\mathcal X}\mu^{s}(d x)  &= \int_{\mathbb{R}^{s}} \mathbbm{1}_{x\in
\mathcal X}\mathbbm{1}_{\|x\|_1\in(0,t]} \prod_{i=1}^s \exp(- \lambda^{-1} |x_i| )\mu^{s}(d x) \\
&\le   \int_{\mathbb{R}^{s}} \prod_{i=1}^s \exp(- \lambda^{-1} |x_i| )\mu^{s}(d x) \\
&=   (2\lambda)^s \\
\ee

\begin{equation}
\label{wass2}
\begin{aligned}
& \underset{g:\|g\|_L\le 1}\sup\int_{\mathbb{R}^{s}}  \mathbb{\| E}(g(\theta) \mid x)\|{ \exp(- \lambda^{-1} \| x\|_1)}   
\mathbbm{1}_{x \neq \bf 0} \mathbbm{1}_{x\in
\mathcal X}\mu^{s}(d x) \\
&\le \underset{g:\|g\|_L\le 1} \sup\quad \underset{x^{*}:\|x^{*}\|_1\in(0,t]\cap
\mathcal X}\sup \|\mathbb{E}(g(\theta) \mid x^*)\|     
\int_{\mathbb{R}^{s}}\mathbbm{1}_{\|x\|_1\in(0,t]}\exp(- \lambda^{-1} \| x \|_1)\mathbbm{1}_{x\in
\mathcal X}\mu^{s}(d x) \\ & + \exp(- \lambda^{-1} t ) \underset{g:\|g\|_L\le 1}\sup \int_{\mathbb{R}^{s}}   \mathbbm{1}_{\|x\|_1\in(t,\infty)}\|\mathbb{E}( g(\theta )\mid x)\| \mathbbm{1}_{x\in
\mathcal X}\mu^{s}(d x) \\
&\le \underset{g:\|g\|_L\le 1}\sup \quad \underset{x^{*}:\|x^{*}\|_1\in(0,t]\cap
\mathcal X}\sup \|\mathbb{E}(g(\theta) \mid x^*)\|     
\int_{\mathbb{R}^{s}}\mathbbm{1}_{\|x\|_1\in(0,t]}\exp(- \lambda^{-1} \| x \|_1) \mathbbm{1}_{x\in
\mathcal X}\mu^{s}(d x) \\ & + \exp(- \lambda^{-1} t ) \underset{g:\|g\|_L\le 1}\sup \int_{\mathbb{R}^{s}}   \mathbbm{1}_{\|x\|_1\in(t,\infty)}\mathbb{E}(\| g(\theta )\|\mid x) \mathbbm{1}_{x\in
\mathcal X}\mu^{s}(d x) \\
&\le \underset{g:\|g\|_L\le 1}\sup \quad\underset{x^{*}:\|x^{*}\|_1\in(0,t]\cap
\mathcal X}\sup\|\mathbb{E}(g(\theta) \mid x^*)\| (2\lambda)^s + \exp(- \lambda^{-1} t ) \underset{g:\|g\|_L\le 1} \sup \mathbb{E}(\| g(\theta )\| \mathbbm{1}_{v(\theta)\in
\mathcal X}) \\
\end{aligned}
\end{equation},

Combining \eqref{wass0}\eqref{wass1}\eqref{wass2}, $k_1= \underset{g:\|g\|_L\le 1}\sup \quad \underset{x^{*}:\|x^{*}\|_1\in(0,t]\cap
\mathcal X}\sup\|\mathbb{E}(g(\theta) \mid x^*)\|$, $k_2=\underset{g:\|g\|_L\le 1} \sup \mathbb{E}(\| g(\theta )\| \mathbbm{1}_{v(\theta)\in
\mathcal X}) $, $k_3= \underset{x^{*}:\|x^{*}\|_1\in(0,t]\cap
\mathcal X}\sup m(x^{*})  $


\begin{equation}
\begin{aligned}
& \underset{g:\|g\|_L\le 1}\sup \int g(x) \Pi(dx) -  \int g(x) \tilde\Pi(dx) \\
        %&\le  \frac{k_1}{ m_0^2} \left[\lambda k_3  + \exp(- \lambda^{-1} t ) \right] 
        %+ \frac{1} {  m_0} [ k_1       \lambda + \exp(- \lambda^{-1} t )k_2 ]\\     
        & \le (2\lambda)^{s} (\frac{k_1 k_3}{m_0^2} + \frac{k_1}{m_0}) + \exp(- \lambda^{-1} t )(\frac{k_1}{m_0^2} + \frac{k_2}{m_0})
        \end{aligned}
        \end{equation}


        \end{proof}

        


The first part shows the asymptotic accuracy of the approximation. The second part shows the rate with non-asymptotic $\lambda$ under mild assumptions. The interpretation for these assumptions is that if in a small space expansion of $\mc D$, defined as $\{\theta^*: \|v(\theta^*)\|_{1}\in [0,t] \}$, the marginal density of $v(\theta^*)$ and the conditional expectation of Lipschitz functions are bounded $k_1,k_2= \mc O(1)$, and the expected norm of Lipschitz function are smaller than a bound that grows near exponentially $k_3 = \mc O(\lambda \exp(t/\lambda))$, then the distance $W_1(\Pi,\tilde\Pi)$ converges to $0$ in $\mc O(\lambda^s)$ as $\lambda\rightarrow 0$.



\bibliography{reference}
\bibliographystyle{chicago}
\end{document}


