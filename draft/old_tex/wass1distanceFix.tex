%&latex
\documentclass[10pt,fleqn]{article}
\pdfoutput=1

\addtolength{\oddsidemargin}{-.875in}
\addtolength{\evensidemargin}{-.875in}
\addtolength{\textwidth}{1.75in}

\addtolength{\topmargin}{-.875in}
\addtolength{\textheight}{1.75in}

\openup 1em

%macro for commenting
\usepackage{color}
\newcommand{\leo}[1]{{\color{blue}{Leo: #1}}}

% \newcommand{\Xbeta}{ X_i \theta}
\newcommand{\xbeta}{ x_i \beta}
\newcommand{\xtheta}{ x_i \theta}
% \newcommand{\xbetaij}{ x_{ij}^T \theta}
\newcommand{\sgamma}{s_{ij}^T\gamma_i}

\usepackage[round]{natbib}

\usepackage{rotating}
\usepackage{graphicx}
\usepackage{subcaption}

\usepackage{float}
\usepackage{bbm}

\usepackage{amsthm,amsmath, amssymb} 
\usepackage{mathrsfs}
\usepackage{subcaption}
\usepackage{nicefrac}

\usepackage{xcolor}
\newcommand{\aki}[1]{\textcolor{red}{Aki: #1}}

\newtheorem{theorem}{Theorem}
\newtheorem{lemma}{Lemma}
\newtheorem{corollary}{Corollary}
\newtheorem{remark}{Remark}
\newtheorem{example}{Example}


\usepackage{algorithm}
\usepackage{algpseudocode}

%\usepackage{mhequ}
\newcommand{\be}{\begin{equation}\begin{aligned}}
\newcommand{\ee}{\end{aligned}\end{equation}}
\newcommand{\bb}[1]{\mathbb{#1}}
\newcommand{\mc}[1]{\mathcal{#1}}
\DeclareMathOperator{\Binom}{Binomial}
\DeclareMathOperator{\No}{No}
\DeclareMathOperator{\PG}{PG}
\DeclareMathOperator{\IG}{Inverse-Gamma}
\DeclareMathOperator{\Ga}{Gamma}
\DeclareMathOperator{\Bern}{Bernoulli}
\DeclareMathOperator{\U}{Uniform}
\DeclareMathOperator{\Poi}{Poisson}
\DeclareMathOperator{\NB}{NB}
\DeclareMathOperator{\cov}{cov}
\DeclareMathOperator{\var}{var}
\DeclareMathOperator{\diag}{diag}
\DeclareMathOperator{\Diag}{Diag}
\newcommand{\KL}[2]{\textnormal{KL}\bigg(#1 \parallel #2\bigg)}
\DeclareMathOperator{\1}{\mathbbm{1}}
\DeclareMathOperator{\bigO}{\mc O}
\newcommand{\dt}{\epsilon} % Stepsize of leapfrog
\newcommand{\mass}{M} % Mass matrix
\newcommand{\hess}{\mathbf{H}} % Hessian notation.



\thispagestyle{empty}
\baselineskip=28pt

\title{\textbf{Constraint Relaxation for Bayesian Modeling with Parameter Constraints}}
\author{Leo Duan,  Alex Young, Akihiko Nishimura, David Dunson}
\date{}
\begin{document}

\maketitle
%\clearpage\pagebreak\newpage
\pagenumbering{arabic}

Let the approximating function for $\{v_{j}(\theta)=0\}_{j=1}^s$ take the form
$\exp(-\frac{ \|v(\theta)\|_1}{\lambda})=\exp(-\frac{ \sum_j|v_j(\theta)|}{\lambda})$, with $\lambda >0$ a scalar. Letting $\bar{\mc H}^{p-s}(d \theta)$ be the $(p-s)$-dimensional
 Hausdorff measure, 
we assume $\mc R$ is chosen so that   $0< m(x)
=\int_{v^{-1}(x)}\frac{\pi_{\mc R}(\theta)}{Jv(\theta)}  \bar{\mc H}^{p-s}(d \theta) <\infty$ for any $x\in \mathcal X =\{v(\theta);\theta\in \mc R\}$
and $\bf 0 \in \mc X$.



 
For any given set $A \subseteq \mc R $, the exact and approximate measures are

\be
\Pi(A) & =\int_{\mathbb R^s} \frac{1}{m_0}\bigg[\int_{v^{-1}({x}) \cap A}\frac{\pi_{\mc R}(\theta)}{Jv(\theta)}  \bar{\mc H}^{p-s}(d \theta)\bigg] \delta_{\bf 0}(x)\mu^{s}(dx)
\\
\tilde\Pi(A) &=\int_{\mathbb R^s} \frac{1}{m_\lambda}\bigg[\int_{v^{-1}(x) \cap A} \frac{\pi_{\mc R}(\theta)}{Jv(\theta)}  \bar{\mc H}^{p-s}(d \theta)
\bigg] \exp(-\frac{\|x\|_1}{\lambda})\mathbbm{1}_{x\in
\mathcal X}\mu^{s}(dx)
\ee
 where $\delta_{\bf 0}(x)$ is a Dirac measure at $\bf 0$; $\mu^{s}(dx)$ is the
 $s$-dimensional Lebesgue measure; $m_0$ and $m_\lambda$ are normalizing constant so that
 $\Pi(\mc R)=1$ and $\tilde\Pi(\mc R)=1$.

Similarly, the expectations of an integrable function
 $g(\theta)$ are
 \be
\bb E_{\Pi}(g(\theta)) & =\int_{\mathbb R^s} \frac{1}{m_0}\bigg[\int_{v^{-1}(x) } g(\theta)\frac{\pi_{\mc R}(\theta)}{Jv(\theta)}  \bar{\mc H}^{p-s}(d \theta)\bigg] \delta_{\bf 0}(x)\mu^{s}(dx)\\
\bb E_{\tilde \Pi}(g(\theta)) &=\int_{\mathbb R^s} \frac{1}{m_\lambda}\bigg[\int_{v^{-1}(x) }g(\theta) \frac{\pi_{\mc R}(\theta)}{Jv(\theta)}  \bar{\mc H}^{p-s}(d \theta)
\bigg] \exp(-\frac{\|x\|_1}{\lambda})\mathbbm{1}_{x\in
\mathcal X}\mu^{s}(dx)
\ee


We denote integral $h(g;x)=\int_{v^{-1}(x) } g(\theta)\frac{\pi_{\mc R}(\theta)}{Jv(\theta)}  \bar{\mc H}^{p-s}(d \theta)$, which is related to
conditional expectation $\bb E(g(\theta)\mid v(\theta)=x) =\frac{1}{m(x)}h(g;x)$.  

 
 
  
\begin{remark}
The 1-Wasserstein distance between the measures $\Pi$ and $\tilde \Pi$ has
$$ \underset{\lambda \to 0}\lim W_1(\Pi,\tilde\Pi)=0.$$
Further,
\begin{equation}
\begin{aligned}
W_1(\Pi,\tilde\Pi) \le  (2\lambda)^{s} (\frac{k_1 k_3}{m_0^2} + \frac{k_1}{m_0}) + \exp(- \lambda^{-1} t )(\frac{k_1}{m_0^2} + \frac{k_2}{m_0}),
\end{aligned}
\end{equation}
where $k_1= \underset{g:\|g\|_L\le 1}\sup \quad \underset{x^{*}:\|x^{*}\|_1\in[0,t]\cap
\mathcal X}\sup |h(g; x^*)|$, $k_2=\underset{g:\|g\|_L\le 1} \sup \mathbb{E}(| g(\theta )| )$ with expectation taken over $\pi_{\mc R}(\theta)$, $k_3= \underset{x^{*}:\|x^{*}\|_1\in(0,t]\cap
\mathcal X}\sup m(x^{*})$.
\end{remark}


\begin{proof}[Proof]

Let $g:\mathbb{R}^p\to \mathbb{R}$ be a 1-Lipschitz continuous function, i.e. $\|g(x)-g(y)\|\le \|x-y\|$, denoted by $\|g\|_L\le 1$. 
By Kantorovich-Rubinstein duality, the 1-Wasserstein distance based on Euclidean metric equals to: 

\begin{equation}
W_1(\Pi,\tilde\Pi)=\underset{g:\|g\|_L\le 1}\sup \bb E_{\Pi}(g(\theta)) -
\bb E_{\tilde \Pi}(g(\theta))
\end{equation}


%The co-area formula from \cite{federer2014geometric} is
% \begin{equation}
% \ \int_{\mathbb{R}^p}  f(\theta)J\Phi(\theta)\mu^p(d \theta)
% =\ \int_{\mathbb{R}^{s}} \bigg[   \int_{\Phi^{-1}(x)}f(\theta)  \bar{\mc H}^{p-s}(d\theta) \bigg]\mu^{s}(d x),
% \end{equation}
 


The two normalizing constants are:

\be
m_0 = &  \int_{\mathbb R^s}m(x) \delta_{\bf 0}(x)\mu^{s}(dx),  \\
m_\lambda  = & \int_{\mathbb R^s} m(x) \exp(-\frac{\|x\|_1}{\lambda})\mathbbm{1}_{x\in
\mathcal X}\mu^{s}(dx).
\ee

Noting $ \delta_{\bf 0}(x) = \lim_{\lambda_0 \to 0}  \exp(-\frac{\|x\|_1}{\lambda_0})$,
we have  $ \exp(-\frac{\|x\|_1}{\lambda_0}) \le  \exp(-\frac{\|x\|_1}{\lambda})$ for any $\lambda \ge \lambda_0$, therefore $m_0 \le m_\lambda$.


We have



\begin{equation}                
\label{wass0}
\begin{aligned}
\underset{g:\|g\|_L\le 1}\sup \bb E_{\Pi}(g(\theta)) -
\bb E_{\tilde \Pi}(g(\theta)) &= \underset{g:\|g\|_L\le 1}\sup \bb E_{\tilde \Pi}(g(\theta)) - \bb E_{\Pi}(g(\theta)) 
\\
=& \underset{g:\|g\|_L\le 1}\sup\int_{\mathbb{R}^{s}}  h(g; x)  \bigg[ \frac{ \exp(- \lambda^{-1} \|x\|_1) } {  m_\lambda}  - 
\frac{ \delta_{\bf 0}(x) } {  m_0} 
\bigg] \mathbbm{1}_{x\in
\mathcal X}\mu^{s}(d x) \\
= &\underset{g:\|g\|_L\le 1}\sup\int_{\mathbb{R}^{s}}  h(g; x) 
 \bigg[ \mathbbm{1}_{x=\bf 0}(\frac{1}{m_\lambda}-\frac{1}{m_0})
 + \mathbbm{1}_{x \in \mc X \setminus\bf 0}(\frac{ \exp(- \lambda^{-1} \|x\|_1) }{m_\lambda})
\bigg]\mu^{s}(d x) \\   
 \le &\underset{g:\|g\|_L\le 1}\sup\int_{\mathbb{R}^{s}}  |h(g; x)| 
 \bigg[ \mathbbm{1}_{x=\bf 0}(\frac{1}{m_0}-\frac{1}{m_\lambda})
 + \mathbbm{1}_{x \in \mc X \setminus\bf 0}(\frac{ \exp(- \lambda^{-1} \|x\|_1) }{m_\lambda})
\bigg]\mu^{s}(d x) \\  
\le &\underset{g:\|g\|_L\le 1}\sup\int_{\mathbb{R}^{s}}  |h(g; x)| 
 \bigg[ \mathbbm{1}_{x=\bf 0}(\frac{1}{m_0}-\frac{1}{m_\lambda})
 + \frac{1}{m_0}\mathbbm{1}_{x \in \mc X \setminus\bf 0}({ \exp(- \lambda^{-1} \|x\|_1) })
\bigg]\mu^{s}(d x)
\end{aligned}
\end{equation}




1. Asymptotic result:






Note $m_\lambda\le  \int_{\mathbb{R}^{s}}  m(x)\mathbbm{1}_{x\in
\mathcal X}\mu^{s}(d x) =\int_{\mathbb{R}^{p}} \pi_{\mc R}(\theta) \mathbbm{1}_{v(\theta)\in
\mathcal X} \mu^p(d \theta) =1$. By dominated convergence theorem, 

\begin{equation}
\lim_{\lambda\to 0}m_\lambda=  \int_{\mathbb{R}^{s}}  m(x)\lim_{\lambda\to 0}\exp(- \lambda^{-1} \| x\|_1 )  \mathbbm{1}_{x\in
\mathcal X}\mu^{s}(d x)   = m_0.
\end{equation}


Since 
\be
&\underset{g:\|g\|_L\le 1}\sup\int_{\mathbb{R}^{s}}  |h(g; x)| 
 \bigg[ \mathbbm{1}_{x=\bf 0}(\frac{1}{m_0}-\frac{1}{m_\lambda})
 + \mathbbm{1}_{x \in \mc X \setminus\bf 0}(\frac{ \exp(- \lambda^{-1} \|x\|_1) }{m_0})
\bigg]\mu^{s}(d x) \\
\le & \int_{\mathbb{R}^{s}} \underset{g:\|g\|_L\le 1}\sup |h(g; x)| 
 \bigg[ \mathbbm{1}_{x=\bf 0}(\frac{1}{m_0}-\frac{1}{m_\lambda})
 + \mathbbm{1}_{x \in \mc X \setminus\bf 0}(\frac{ \exp(- \lambda^{-1} \|x\|_1) }{m_0})
\bigg]\mu^{s}(d x),
\ee
letting $q_\lambda=  \underset{g:\|g\|_L\le 1} \sup |h(g; x)| 
 \bigg[ \mathbbm{1}_{x=\bf 0}(\frac{1}{m_0}-\frac{1}{m_\lambda})
 + \mathbbm{1}_{x \in \mc X \setminus\bf 0}(\frac{ \exp(- \lambda^{-1} \|x\|_1) }{m_0})
\bigg]$, for fixed $x$, a decreasing series $1\ge\lambda_1\ge \lambda_2\ge \ldots$, has monotone increasing $0\le q_1-q_{\lambda_1}\le q_1-q_{\lambda_2}$ , by monotone convergence theorem, $\lim_{\lambda\to 0}\int [ q_1(x)-q_\lambda(x)]dx = \int [q_1(x)- q_0(x) ]dx$ hence $\lim_{\lambda\to 0}\int q_\lambda(x)dx =0$. Combining the results yields 



\begin{equation}
\underset{\lambda \to 0}\lim W_1(\Pi,\tilde\Pi)=0.        \end{equation}

2. Non-asymptotic result:


\begin{equation}
\label{wass1}
\begin{aligned}
\frac{1}{m_0}-\frac{1}{m_\lambda} 
& \le \frac{m_\lambda -m_0}{m_0^2} \\ 
\\ &=  \frac{    \int_{\mathbb{R}^{s}}  m(x)\exp(- \lambda^{-1} \| x\|_1 )\mathbbm{1}_{x\in
\mathcal X \setminus \bf 0}\mu^{s}(d x) } {  m_0^2}  \\
&= \frac{1}{ m_0^2} \bigg[ \int_{\mathbb{R}^{s}} \mathbbm{1}_{\|x\|_1\in(0,t]}  m(x) \exp(- \lambda^{-1} \| x\|_1  )\mathbbm{1}_{x\in
\mathcal X}\mu^{s}(d x) + \int_{\mathbb{R}^{s}}
 \mathbbm{1}_{\|x\|_1\in(t,\infty)}  m(x) \exp(- \lambda^{-1} \| x\|_1  )
 \mathbbm{1}_{x\in
\mathcal X} \mu^{s}(d x) \bigg] \\
&\le \frac{1}{ m_0^2} \bigg[  \underset{x^{*}:\|x^{*}\|_1\in(0,t]\cap
\mathcal X}\sup m(x^{*}) \int_{\mathbb{R}^{s}}\mathbbm{1}_{\|x\|_1\in(0,t]}  \exp(- \lambda^{-1} \|x\|_1 )\mathbbm{1}_{x\in
\mathcal X}\mu^{s}(d x) \\ &+ \exp(- \lambda^{-1} t) \int_{\mathbb{R}^{s}}\mathbbm{1}_{\|x\|_1\in(t,\infty]}  m(x)\mathbbm{1}_{x\in
\mathcal X}\mu^{s}(d x)   \bigg] \\
&\le \frac{1}{ m_0^2} \bigg[(2\lambda)^s  \underset{x^{*}:\|x^{*}\|_1\in(0,t]\cap
\mathcal X}\sup m(x^{*})  + \exp(- \lambda^{-1} t ) \bigg] 
\end{aligned}
\end{equation}
where 
\be
\int_{\mathbb{R}^{s}}\mathbbm{1}_{\|x\|_1\in(0,t]}  \exp(- \lambda^{-1} \|x\|_1 )\mathbbm{1}_{x\in
\mathcal X}\mu^{s}(d x)  &= \int_{\mathbb{R}^{s}} \mathbbm{1}_{x\in
\mathcal X}\mathbbm{1}_{\|x\|_1\in(0,t]} \prod_{i=1}^s \exp(- \lambda^{-1} |x_i| )\mu^{s}(d x) \\
&\le   \int_{\mathbb{R}^{s}} \prod_{i=1}^s \exp(- \lambda^{-1} |x_i| )\mu^{s}(d x) \\
&=   (2\lambda)^s \\
\ee

\begin{equation}
\label{wass2}
\begin{aligned}
& \underset{g:\|g\|_L\le 1}\sup\int_{\mathbb{R}^{s}} |h(g; x)|   { \exp(- \lambda^{-1} \| x\|_1)}   
\mathbbm{1}_{x \neq \bf 0} \mathbbm{1}_{x\in
\mathcal X}\mu^{s}(d x) \\
&\le \underset{g:\|g\|_L\le 1} \sup\quad \underset{x^{*}:\|x^{*}\|_1\in(0,t]\cap
\mathcal X}\sup |h(g; x^{*})|     
\int_{\mathbb{R}^{s}}\mathbbm{1}_{\|x\|_1\in(0,t]}\exp(- \lambda^{-1} \| x \|_1)\mathbbm{1}_{x\in
\mathcal X}\mu^{s}(d x) \\ & + \exp(- \lambda^{-1} t ) \underset{g:\|g\|_L\le 1}\sup \int_{\mathbb{R}^{s}}   \mathbbm{1}_{\|x\|_1\in(t,\infty)}|h(g; x)| \mathbbm{1}_{x\in
\mathcal X}\mu^{s}(d x) \\
&\le \underset{g:\|g\|_L\le 1}\sup \quad \underset{x^{*}:\|x^{*}\|_1\in(0,t]\cap
\mathcal X}\sup |h(g; x^{*})|     
\int_{\mathbb{R}^{s}}\mathbbm{1}_{\|x\|_1\in(0,t]}\exp(- \lambda^{-1} \| x \|_1) \mathbbm{1}_{x\in
\mathcal X}\mu^{s}(d x) \\ & + \exp(- \lambda^{-1} t ) \underset{g:\|g\|_L\le 1}\sup \int_{\mathbb{R}^{s}}   \mathbbm{1}_{\|x\|_1\in(t,\infty)}h(|g|; x) \mathbbm{1}_{x\in
\mathcal X}\mu^{s}(d x) \\
&\le \underset{g:\|g\|_L\le 1}\sup \quad\underset{x^{*}:\|x^{*}\|_1\in(0,t]\cap
\mathcal X}\sup|h(g; x^{*})|(2\lambda)^s + \exp(- \lambda^{-1} t ) \underset{g:\|g\|_L\le 1} \sup \mathbb{E}( | g(\theta )| \mathbbm{1}_{v(\theta)\in
\mathcal X}), \\
\end{aligned}
\end{equation}
where we used $h(|g|;x)=\int_{v^{-1}(x) } |g(\theta)|\frac{\pi_{\mc R}(\theta)}{Jv(\theta)}  \bar{\mc H}^{p-s}(d \theta) \ge h(g;x)$ and $ \int_{\mathbb{R}^{s}}   h(|g|; x) \mathbbm{1}_{x\in
\mathcal X}\mu^{s}(d x) = \bb E(|g(\theta)|) $  with expectation taken over $\pi_{\mc R}(\theta)$.


Combining \eqref{wass0}\eqref{wass1}\eqref{wass2}, $k_1= \underset{g:\|g\|_L\le 1}\sup \quad \underset{x^{*}:\|x^{*}\|_1\in[0,t]\cap
\mathcal X}\sup |h(g; x^*)|$, $k_2=\underset{g:\|g\|_L\le 1} \sup \mathbb{E}(| g(\theta )| )$, $k_3= \underset{x^{*}:\|x^{*}\|_1\in(0,t]\cap
\mathcal X}\sup m(x^{*})$


\begin{equation}
\begin{aligned}
& \underset{g:\|g\|_L\le 1}\sup \bb E_{\Pi}(g(\theta)) -
\bb E_{\tilde \Pi}(g(\theta)) \\
        %&\le  \frac{k_1}{ m_0^2} \bigg[\lambda k_3  + \exp(- \lambda^{-1} t ) \bigg] 
        %+ \frac{1} {  m_0} [ k_1       \lambda + \exp(- \lambda^{-1} t )k_2 ]\\     
\le & \underset{g:\|g\|_L\le 1}\sup\int_{\mathbb{R}^{s}}  |h(g; x)| 
 \bigg[ \mathbbm{1}_{x=\bf 0}(\frac{1}{m_0}-\frac{1}{m_\lambda})
 + \mathbbm{1}_{x \in \mc X \setminus\bf 0}(\frac{ \exp(- \lambda^{-1} \|x\|_1) }{m_0})
\bigg]\mu^{s}(d x)\\
\le & \underset{g:\|g\|_L\le 1}\sup |h(g; {\bf 0})| 
 (\frac{1}{m_0}-\frac{1}{m_\lambda}) 
 + \underset{g:\|g\|_L\le 1}\sup \int_{\mathbb{R}^{s}} |h(g; x)|  \mathbbm{1}_{x \in \mc X \setminus\bf 0}(\frac{ \exp(- \lambda^{-1} \|x\|_1) }{m_0})
\bigg]\mu^{s}(d x)\\
\le & k_1  \frac{1}{ m_0^2} \bigg[(2\lambda)^s  k_3  + \exp(- \lambda^{-1} t ) \bigg] + \frac{1}{m_0} \bigg [ k_{1}(2\lambda)^s  + \exp(- \lambda^{-1} t ) k_2 \bigg] \\
         \le & (2\lambda)^{s} (\frac{k_1 k_3}{m_0^2} + \frac{k_1}{m_0}) + \exp(- \lambda^{-1} t )(\frac{k_1}{m_0^2} + \frac{k_2}{m_0})
        \end{aligned}
        \end{equation}


        \end{proof}

        


The first part shows the asymptotic accuracy of the approximation. The second part shows the rate with non-asymptotic $\lambda$ under mild assumptions. The interpretation for these assumptions is that if in a small space expansion of $\mc D$, defined as $\{\theta^*: \|v(\theta^*)\|_{1}\in [0,t] \}$, the marginal density of $v(\theta^*)$ and the integral of Lipschitz functions are bounded $k_1,k_2= \mc O(1)$, and the expected norm of Lipschitz function are smaller than a bound that grows near exponentially $k_3 = \mc O(\lambda \exp(t/\lambda))$, then the distance $W_1(\Pi,\tilde\Pi)$ converges to $0$ in $\mc O(\lambda^s)$ as $\lambda\to 0$.



\bibliography{reference}
\bibliographystyle{chicago}
\end{document}


